\section[cert]{Formalizing Entity Resolution Comparison}\label{subsec:cert}

Entity resolution tasks can be compared in quantitative terms without any
issue.
Metrics for the amount of hardware resources they consume or the physical
time that elapses for a data set of a given size are invariant across the
many possible implementations of entity resolution solutions.

The problem of comparing entity resolution results from a qualitative
perspective is significantly less clearly delimited.
Each mathematical model is rooted in a different branch of mathematics and,
as such, works with different assumptions and different expressions of the
surrounding reality.
For example, all statistical models express the real world in terms of
probabilities.
On the other hand, a construct of set theory will invariably explain the
real world in terms of set operations, properties and it will use algebraic
constructs.
The comparison problem must take into account these limitations.

\begin{defn}
    Given
    \begin{itemize}
        \item the set of entity references $Ref$,
        \item a set of quality metrics $Q$ that are common to all
        entity resolution tasks $e_i : Ref -> Res$, $i \in \mathbb{N}$
        \item the ground truth $G$ for the input domain $Ref$,
        \item the quality metric results
        $V_i=\{q_{i} \mid q_{i} = q(G, e_i), q \in Q\}, i \in \mathbb{N}$,
    \end{itemize}   
    we define the comparison of the entity resolution tasks above as
    $C : (
        \{V_i \mid i \in \mathbb{N}\}
        \times
        \{V_j \mid j \in \mathbb{N}\},
        i \neq j
    ) \rightarrow \{-1, 0, 1\}$ such that:
    \[ 
C(V_i, V_j) = \left\{
\begin{array}{ll}
  -1,~\textrm{if $e_i$ is less fit for purpose than $e_j$}\\
  0,~\textrm{if $e_i$ is as fit for purpose as $e_j$}\\
  1,~\textrm{if $e_i$ is better fit for purpose than $e_j$}\\
\end{array} 
\right. 
\]

\end{defn}

The difficulty with the above definition stems from the formulation around
quality metrics.
One cannot compare variation of information with statistical precision, for
example.
Therefore, the metrics that are used in the comparison must be of a kind.
And we know that each type of metric is determined by a mathematical model.
