\documentclass[journal]{IEEEtran}

% Packages
\usepackage{amsmath,amsfonts,amsthm}
\usepackage{algorithmic}
\usepackage{balance}
\usepackage[caption=false,font=normalsize,labelfont=sf,textfont=sf]{subfig}
\usepackage{enumitem}
\usepackage{hyperref}
\usepackage{xcolor}
\usepackage{cite}
\usepackage{array}
\usepackage{graphicx}
\hyphenation{op-tical net-works semi-conduc-tor IEEE-Xplore}
\graphicspath{ {./img/} }
\newcommand{\BibTeX}{\textrm{B \kern-.05em \textsc{i \kern-.025em b} \kern-.08em
T \kern-.1667em \lower.7ex \hbox{E} \kern-.125emX}}
\usepackage{balance}

% Document
\begin{document}
    \title{A Tale of Two Entity Resolution Models}
    % \author{Andrei Olar \textcolor{green}{eu m-as bucura sa ma incluzi si pe mine; daca este de acord poti formata ca mai jos partea de autori}}
    \author{
        \IEEEauthorblockN{
            Andrei Olar\IEEEauthorrefmark{1},
            Laura Dio\c san\IEEEauthorrefmark{2}
        }
        \IEEEauthorblockA{
            Faculty of Mathematics and Computer Science, Babe\c s-Bolyai University\\
            Email: 
                \IEEEauthorrefmark{1}andrei.olar@ubbcluj.ro,
                \IEEEauthorrefmark{2}laura.diosan@ubbcluj.ro
        }
    }

    \maketitle

    \theoremstyle{definition}
    \newtheorem{defn}{Definition}[section]
    
    \maketitle
    \begin{abstract}
        This paper examines entity resolution, the process of identifying if two
        information represent the same real-world entity.
        Despite well-established theoretical models in this field, challenges
        persist in selecting appropriate entity resolution tasks for specific
        contexts.
        These models are associated with quality metrics that assess the
        outcomes of entity resolution tasks.
        Our study demonstrates the possibility of identifying abstract
        conditions invariant to specific solutions and data sets, which instead
        relate to the theoretical model used for evaluating entity resolution
        outcomes.
        This approach offers a novel perspective for understanding the
        performance of entity resolution tasks.
    \end{abstract}

    \begin{IEEEkeywords}
        Entity Resolution, Fellegi-Sunter, Algebraic Model, Metrics, Evaluation
    \end{IEEEkeywords}

    \section{Introduction}\label{sec:introduction}
    Entity resolution is the task of determining whether two information refer
    to the same real-world item or not.
    More restrictive definitions place entity resolution as the task of
    identifying and linking representations of data from two or more
    sources~\cite{Qia17}.
    However, we share the opinion that identifying and linking data constitutes
    a more specialized process~\cite{Tal11}.
    The task of gathering information about a generic pound of potatoes across
    various marketplaces is still an entity resolution task, in our view.
    Whether the information about marketplaces and the potatoes that find
    themselves passing through there is available in one or more places strikes
    us as non-consequential, too.
    Besides not talking about identity problems, we also leave aside any
    classification of entity resolution algorithms based on number of sources.
    
    Precisely because it is a fundamental problem, entity resolution goes by
    many names: record linkage, data de-duplication, merge-purge, named entity
    recognition and disambiguation (or NERD), entity alignment or entity
    matching~\cite{Tal11,fever2009,alhelbawy2014}.
    The number of names is matched by the number of practical applications for
    entity resolution, ranging from linking medical records to assisting in
    medical diagnostics, to translation, to doing background checks for
    financial crime or to identifying plagiarism.
    Combining information from different media (sound, images, motion, smell,
    etc.) can also be regarded as a form of entity resolution when the purpose
    is to gather more information about the same real-world object.
    With such a wide array of past, present and future applications for entity
    resolution, evaluating its quality seems important.

    With the current paper, our goal is to introspect how entity resolution
    theoretical models influence our perception over entity resolution results.
    The discussion is around choosing entity resolution solutions that are
    more or less suitable based on how we perceive their outcomes.
    In this context, we are interested in how entity resolution quality metrics
    complement one another.
    Partly in order to answer this last question, we are looking into the
    diversity of the insights provided by different entity resolution models
    regarding the same entity resolution result.
    
    We begin with a preliminary description focused on further detailing the
    goals of this paper.
    After briefly summarizing the work that relates to our paper, we introduce
    terminology and a mental model that we use throughout the remaining
    sections.
    Afterwards, we revisit the two theoretical models for entity resolution
    chosen for our analysis.
    The text describing theoretical models is accompanied by experiments using
    an entity resolution solution of choice for which we present our findings.
    We discuss our observations, pointing out invariant conditions that do not
    seem to be related to the choice of data set.

    \section[preliminaries-questions]{Preliminaries \& Research Questions}
    \label{sec:preliminaries-questions}
    In scientific literature, entity resolution has been attributed numerous
characteristics\cite{Tal11,Pap19}.
In the context of this paper, the following stand out:

\begin{itemize}
    \item\textit{A concern for a real-world entity}, for if the task were
    not about a something from our world, there couldn't be any resolution;
    \item\textit{No implication as to the used method}, because if we
    stipulated a certain way to perform entity resolution, we might not be
    able to;
    \item\textit{The representation of information is computer friendly},
    because if it weren't, it might not be possible to study the task by
    means of computer science.
\end{itemize}

The first two statements suggest that entity resolution can be performed
using any informational support.
Simply imagine collecting written notes about the same real-world object to
create an informational profile of that object.
Entity resolution becomes relevant to computer science when it involves
processing information about real-world entities through computational
methods, rather than directly interacting with the entities
themselves~\cite{Tal11}.
This approach is characterized by the constraint of handling data that is
computable, indicating that entity resolution tasks executed on computers
do not have direct access to the entities they aim to resolve~\cite{Chen09}.

Whenever we talk about measuring the performance of an entity resolution
task, we must note that we are confined to the information that is available
to the computer running the task.
On the other hand, as humans we judge the quality of the entity resolution
task's result in relation to the real-world information available to us, which
usually includes the real-world object itself.
This produces a gap between what a computer can provide us with and the human
observer's expectation, a gap which is starting to garner attention~\cite{wang2022realideal}.
Coincidentally we are nearing a century since it has become apparent that the
observer is key in our physical universe~\cite{schrodinger1926}.
Perhaps we can find inspiration for understanding entity resolution quality in
Schrodinger's insight.

We notice that there are multiple models that describe entity resolution.
Each of these models uses the tools that are available to it as part of its
theoretical foundation.
Some use probabilities, some use graphs and some use sets and linear algebra.
Each of them presents the human user with a unique perspective over how entity
resolution functions, a perspective that includes the presentation of the
outcomes of entity resolution.
On the other hand, these models are generic: all entity resolution models apply
to all entity resolution solutions.

In this context, we may find conditions that originate in the entity resolution
model itself rather than in reality, the data provided to the computer or the
entity resolution solution.
These conditions might be abstract, but they might also provide valuable
insight to the human observer that analyses an entity resolution outcome.
Our \textbf{first goal is to find out whether such invariant conditions might
actually exist}.

Next, another gap presents itself between the generic solutions provided by
scientific literature or programming libraries and the specific needs derived
from a context of the practitioners actually implementing entity resolution
systems.

Historically, entity resolution measurements owe much to the field of
information retrieval and to statistics.
Implementing entity resolution systems in practice, we often notice that
using popular metrics borrowed from these fields (such as the $F_1$ score) we
end up overlooking some essential aspect.
For example, in anomaly detection problems we are primarily interested in
sensitivity, not the balanced outlook the $F_1$ score provides.
Given that anomaly detection represents a whole class of solutions, one might
argue that the need of highly sensitive systems is not that specific.
Which makes this point all the more relevant.

In scientific literature and programming, generic solutions are preferred to
specific ones.
This bias towards generic solutions makes it hard for practitioners to rank
entity resolution solutions by their usefulness in the context of the task
at hand (e.g anomaly detection).
Choosing an entity resolution solution remains a necessary step in
implementing an entity resolution system.
The longer it takes to find one, the higher the implementation costs.
Also, choosing a complex system is often a decision that's difficult to
change later and this contributes to the risk of making a poor choice.

Papers have been published on the biases inherent in certain metrics
and how they influence judging entity resolution outcomes in a specific
context~\cite{Goga2015}.
The few systems out there that do provide some form of ranking entity
resolution solutions tend to use balanced metrics~\cite{papwithcode2019}.
Our previous example shows that being balanced is actually just another bias
which may or may not be useful, depending on context.

All this goes to show that the myriad ways~\cite{hitesh2012} in which we
evaluate entity resolution quality suggests that we haven't found the ideal
perspective for the entity resolution problem.
It might also show that it will take us a long time to find it.
From this point of view, it seems logical to look at \textbf{how some of the
existing metrics might complement one another to enhance our view over an entity
resolution result}.

As previously stated, entity resolution quality evaluation owes much to
information retrieval and to probability theory.
Additionally, metrics originally designed to aid in graph theory problems have
become useful, too~\cite{hitesh2012,Kon19}.
Nevertheless, a cursory review of what is usually being measured in lieu of
judging entity resolution quality~\cite{fever2009,Men10,Goga2015} shows an
important interest for evaluating the quality of entity matching.
At this point we refer to entity matching as simply finding out how similar two
data are by using various comparison techniques.
This may be due to a lingering confusion between entity matching and entity
resolution~\cite{Tal11}.

Literature~\cite{Tal11} describes other perspectives available to us regarding
entity resolution, besides seeing it as a matching problem.
The entity resolution quality metrics that measure aspects which go beyond
matching~\cite{Men10,tal2007algebraic} provide us with the pragmatic means to
see these alternate views.
On the other hand, entity resolution models\cite{Ben2009Swoosh,Tal11,fs1969}
provide us with the abstract perspective.
We interpret this to mean that certain metrics are linked to certain entity
resolution models.
With this view in mind, our final question is \textbf{whether it's possible to
interpret the same entity resolution result using two different entity
resolution models}.


    \section{Related Work}\label{sec:related}
    
    The topic addressed in this paper intersects with a significant body of
    existing research.
    Specifically, it aligns closely with three primary categories: presentations
    of entity resolution systems, theoretical model syntheses for entity
    resolution, and analyses of evaluation metrics in the field of entity
    resolution.

    Papers introducing entity resolution systems are relevant to this study due
    to their common objective of standardizing entity resolution tasks and their
    evaluation methods.
    The majority of the referenced systems in this context offer several generic
    metrics for assessing entity resolution performance.

    At the dawn of the 21st century AD, FEBRL~\cite{febrl2002} emerged as one of
    the first extensible systems to tackle measuring entity resolution
    performance.
    It was succeeded by FEVER~\cite{fever2009} and OYSTER~\cite{oyster2012}.
    These systems have all paved the way to generalizing our understanding about
    entity resolution and to ever more improvements in the quality of entity
    resolution tasks.
    Later on, systems like `Papers with Code'~\cite{papwithcode2019} brought in
    a social dimension to how we compare the qualitative performance of entity
    resolution tasks.
    The systems that came after, resemble frameworks more than they resemble
    applications~\cite{magellan2020,jedai2017}.
    Almost all systems are open-source and encourage social collaboration around
    entity resolution, too.
    
    Next, the syntheses on the theoretical models for entity
    resolution~\cite{fs1969,Ben2009Swoosh,Tal11} are invaluable resources.
    The models we present in this paper have been heavily documented
    elsewhere~\cite{Tal11,tal2013}.    
    We overview the models here to fill potential gaps in the reading flow of
    the paper more than to address another need.
    
    Taking a different tack on describing entity resolution in a generic,
    abstract way, there are studies that shine a procedural light over entity
    resolution~\cite{Pap19,Chen09}.
    These studies showcase the steps of the entity resolution process and the
    significance, challenges and benefits of each step.
    These studies show us that matching and clustering (additionally to being
    aspects of entity resolution) are steps in the entity resolution process.

    Lastly, we note syntheses that present numerous entity resolution metrics
    and their application~\cite{hitesh2012,graf2021frost,barnes2015practioner}.
    The current paper is very similar to these syntheses because it, too
    enumerates metrics and it, too attempts to use those metrics to gather
    deeper insights that would alleviate the burden of adopting entity
    resolution.
    We use this work to find the strongest associations of certain metrics with
    an entity resolution model.
    However, our approach is to discern between theoretical models and attempt
    to extract invariant conditions rooted in the model instead of simply using
    the metrics to evaluate the entity resolution result.

    \section{Theoretical background}
    \subsection[terminology]{Terminology pre-requisites}\label{sec:terminology}
    Throughout this paper the term ``entity'' is reserved for real-world items.
We need new terms for the computer-friendly representations of entities.
We start by saying that entity resolution operates on data sources.

\begin{defn}
    A \textit{data source} represents a sequence of decoded messages
    that originate somewhere and can be processed by a computer program that
    reads them over a communication channel\footnote{The terms `decoded', `message' and `communication channel' refer to concepts
    defined in information theory~\cite{ash2012it}.}.
\end{defn}



An entity resolution task needs at least one data source to operate on.
When the relevant information is split across multiple data sources, these
can be merged into a single source.
In the same way, a single data source can be split into multiple data
sources.
Data sources may be bounded (files or databases) or unbounded (streams).
For the purposes of this paper, these attributes of data sources as well as
their number are inconsequential.

In practical terms, entity resolution is a computer process that operates
over a data source.
In the process of entity resolution, we use specific notions to refer to the
various levels in which data is organised.

\begin{defn}
    An \textit{attribute} is information about an entity that has a certain
    meaning in a context given by a frame of reference and certain rules of
    interpretation.
\end{defn}

Attributes by themselves are not enough to describe an entity.
For example, `red' fully describes an entity only if we are looking for
colors.
If we are looking for fruit, `red' is only one of several attributes that
help describe fruit.
We might be interested whether the fruit is `sour' or `sweet' as well as
whether it is `small', `medium' or `large'.

\begin{defn}
    An \textit{entity reference} is a collection of attributes that refer
    to a real-world entity which can be formed by following an organizing
    principle (or order) of the data source where the attributes are all
    located.
\end{defn}

The organizing principle of a data source is simply an order that
facilitates processing information in that data source.
A CSV file has the organizing principle that attributes are separated by
commas.
Relational databases organize data in tables, columns and rows.
Other storage organizes data in structured, semistructured or even
unstructured ways (such as text).
As an example of unstructured organization, named entity disambiguation uses
the same organizing principle as named entity recognition built around
mentions.
When an entity resolution task processes a data source, it loads
\textit{references}\cite{Ben2009Swoosh} to entities by taking advantage of
the way that data source is organized.

We shouldn't confuse the organizing principle of a data source with its
structure, though.
A table record in a database might store multiple attributes of the same
entity in the same column.
The entity reference will be constructed by extracting all attributes, not
by dumbly considering each column to be an attribute.
Entity reference extraction is also selective by nature.
For entity resolution tasks that value geolocation above other things, the
extracted references might skip over data that is not about geography.
Similarly, in tasks that identify people, entity references will be
comprised from attributes that are useful in identifying a person or a
company.
All this speaks to how the organizing principle of a data source is
extrinsic to the data source and linked to the entity resolution's purpose.

A final note on entity references is that attributes that each reference is
bound to a data source.
The definition above states this by mandating that all attributes comprising
an entity reference are read from the same data source.

So far we have dealt with structure, now it is time to deal with meaning and
purpose.    
Let's suppose that we are gathering information about colors.
A data source represents colors as `red', `green' or `blue'.
Another one represents them as \texttt{0xCC0000}, \texttt{00CC00} or
\texttt{0x0000CC}.
The organizing principle instructs the entity resolution task to construct
entity references with a single attribute.
It does not provide any semantic interpretation rules for the data.

\begin{defn}
    A \textit{trait} is a semantic rule that guides the entity resolution
    task in recognizing the organizing principle of each information source
    that it uses as input.
\end{defn}

Each entity resolution task is defined by specific objectives it aims to
achieve.
For every distinct task, there are key rules that guide the interpretation
of the available data, emphasizing certain aspects over others.
In our example, entity resolution focuses exclusively on identifying colors
in different data sources.
To achieve this goal, the traits of the entity resolution task instance
might be specified, trained or inferred from a knowledge base to recognize
various representations of colours.
While we might think of traits as being a part of a generic data extraction
process such as scraping, traits in the sense described here are actually
integral to the entity resolution process itself.

The notion that is the most similar to that of a trait is the pattern
recognition notion of a `feature':
\textit{an individual measurable property or characteristic of a
phenomenon}\cite{bishop2006pattern}.
The mechanics of how traits and features function is near identical.
The difference between a trait as defined here and a feature is a matter of
perspective.
Features are part of constructing a broader, objective perspective on
information.
Traits as they are defined here are concerned with the contextual meaning.

Pragmatically, traits can be viewed as algorithms that extract information
according to the subjective goal of the entity resolution task.
In their simplest form, traits might be configuration parameters for an
entity resolution system.
But they can also be algorithms that are part of the a system's code.

For example, a trait called `profitable' that is supposed to specify whether
a company earns more than it spends is different from the typical physical
representation of profit and loss as two separate decimal values present
in information organized in some way or another.
The way we would handle this discrepancy while extracting entity references
is by having an algorithm that outputs a truth value when gains exceed
spending.

Another example of a trait, is a configuration value that specifies the
attributes that should be used when comparing two entity references.
This trait might optimize certain attributes of the entity references for
comparison.

To avoid confusing traits and attributes, it is easy to think about traits
as not being a part of the data source, but rather as a part of the entity
resolution task implementation.
Entity resolution tasks extract attributes that they assign to entity
references based on the traits specific to each task.

Traits are a necessary abstraction if we want the same entity resolution
task to be able to process more than one data representation.
They are useful for constructing individual entity references.
However, we have not defined any notion that links entity references to an
entity.

\begin{defn}
    The logical group of entity references that point to the same real-world
    entity according to the entity resolution task's parameters is known as
    the \textit{entity profile} of that real-world entity within the entity
    resolution task.
\end{defn}

It is important to understand that the entity profile and the real-world
entity are not the same.
Entity profiles are simply digital collections of entity references that are
formed because of how an entity resolution task was programmed and
configured.

By using the terms \textit{information source}, \textit{attribute},
\textit{trait}, \textit{entity reference} and \textit{entity profile} we can
finally move on to define entity resolution formally.

    
    \subsection{Formalizing Entity Resolution}\label{sec:entity-resolution}
    In a broad sense, the entity resolution process is usually comprised of four
different steps\cite{Pap19,Tal11}:

\begin{itemize}
    \item \textit{entity reference extraction} --- the first phase of entity
    resolution deals with extracting entity references from multiple
    sources, as detailed in Section~\ref{sec:terminology};
    \item \textit{blocking} and \textit{filtering} --- at this stage, the
    entity resolution task groups in the same `block' entities that share
    similar traits and then within each block all entities that cannot
    possibly match are filtered \textit{out}; the intent here is to allow
    entity resolution tasks to work with large amounts of data~\cite{Pap19};
    \item \textit{matching} --- one of the defining activities of the entity
    resolution process, it consists of comparing entity references to one
    another;
    \item \textit{clustering} --- the other defining entity resolution
    activity which collects matching references in entity profiles.
\end{itemize}

In this paper, we concern ourselves with the last two stages of the entity
resolution process because the quality of the process' outcome largely
depends on these steps.
From now on, when we use the term \textit{entity resolution} we only refer
to matching and clustering steps.

However, because extraction plays a big part in how data is structured
within any entity resolution process, we start specifying the entity
resolution problem by formalizing the data extraction using the notions from
Section~\ref{sec:terminology}.

Given:
\begin{itemize}
    \item $n \in N^*$ pairs ($S_i$, $n_i$), $1 \leq i \leq n$, $n_i \in N$, 
    where $S_i$ is a data source and $n_i$ is the number of messages in
    $S_i$,
    \item $m_i \in N^*$ traits $t_{ij}$, $1 \leq j \leq m_i$ for each
    information source $S_i$ so that
    \item each trait $t_{ij}$ can generate at most $a_{ij} \in N^*$
    attributes,
\end{itemize}
an entity resolution task $ER$ applies each trait $t_{ij}$ onto the source
$S_i$, $1 \leq i \leq n$, $1 \leq j \leq m_i$, in order to extract
$r_{{S_i}k}$ entity references where
$0 \leq k \leq n_i \cdot \sum^{m_i}_{j=1}a_{ij}$.

Next, we denote with $A$ the domain of all possible attributes that can be
used by $ER$.

In this context, let $r_{{S_i}{k_j}}$ denote the $k_j$th, $1 \leq j \leq k$
entity reference extracted from the source $S_i$.
Then $r_{{S_i}{k_j}}$ is a tuple
\[
    (a_r, r \in N^*, a_r \in A), r_{{S_i}{k_j}} \in A^r
\]
where $r$ varies for each $k_j$.

Defining traits and the organizing principle as external to the data source
is what allows us to formalize entity references so simply.
A trait is then a function that helps construct entity references.
The definition of traits materializes only upon extracting entity references
$r_{{S_i}{k_j}}$ with $ER$.
The domain of this function is the domain of all attributes in the source
$S_i$.
We denote this domain with $A_{{S_i}{k_j}}$.

A trait can be declared as a function:

\[
    t: S_i \rightarrow A_{{S_i}{k_j}}^x\textrm{, where }x \in N^*
\]
and $x$ varies with $k_j$.

During the extraction process, \textit{ER}'s traits lead to the construction
of the domain of entity references:

\[
    Ref = \bigcup_{i \in N^*,1 \leq j \leq k} r_{{S_i}{k_j}}
    \textrm{, with } r \textrm{, }S \textrm{ and } k \textrm{ specified above.}
\]

The $Ref$ domain is constructed through entity reference extraction.
After the entities have been extracted, we are no longer concerned with the
input information sources.
Every type of entity resolution task can be made to work on such an input
domain~\cite{Pap19}.
De-duplication tasks that use a single data source work out of the box.
Record linking tasks that use two or more data sources can embed an
attribute that identifies the data source of each entity reference.
Regardless of the optics, the input domain contains only entity references.
The relevance of the original data source of an entity reference rests with
the entity resolution task and its matching and clustering algorithms.

The output of entity resolution is always a collection of entity profiles.
While the data representation of an entity profile is definable only per
theoretical model, we can define the output domain of entity resolution

\[
    Res=\{P_x \mid x \in \mathbb{N^*}, P_x \in P\}
\]

\noindent
as the domain of collections of entity profiles, where \textit{P} represents the
domain of entity profiles.
The domain of entity profiles is specified for each entity resolution model.

This specification, finally, allows us to formalize entity resolution itself as a function:

\[
    e(r_1, r_2, \ldots, r_n): Ref^n \rightarrow Res\text{, where }n\in\mathbb{N^*}
\]

\noindent
which takes in a sequence of entity references and outputs a sequence of
entity profiles.

Note that $Res$ is always going to be a collection containing unique entity
profiles in any theoretical model of entity resolution because of the nature
of the entity resolution problem.
Entity resolution disambiguates information, making it impossible for
identical items to be a part of a correct output.
This should not be taken to mean that the output of entity resolution tasks
will always be a set of profiles.

If $Res$ is the domain of all collections of entity profiles, one of
the collections that are part of $Res$ must be the collection of entity
profiles that resembles our understanding of reality the most.

\begin{defn}
    We define the \textit{ground truth} \textbf{G} over the entity reference
    domain \textit{Ref} as the entity resolution result comprised of entity
    profiles:
    \[
        \{P_g \mid g \in \mathbb{N^*}\}\textrm{, such that:}
    \]
    \begin{itemize}
        \item each $P_g$ describes an entity completely using the available
        entity references from $Ref$, and
        \item there are no entity references in \textit{Ref} which are not a
        part of a $P_g \in G$.
    \end{itemize}
\end{defn}

In other words the ground truth is the ideal entity resolution result for 
the given input data, in a context, from a perspective.
Knowing this, we can formalize entity resolution quality generically with
respect to the ground truth.

\begin{defn}
Given an entity resolution \textit{e} and the ground truth \textit{G} for
the entity reference domain $Ref$, a \textit{quality metric} is an
idempotent function:
\[
    q: Res^2 \rightarrow \mathbb{R}
\]
which scores how similar the result of \textit{e} is to \textit{G}.
\end{defn}

In practice, we should strive to implement quality metric as \textit{pure}
functions, guarding their execution from any side-effects specific to
running a program on a computer.

As stated before, the shape and structure of the items in $Res$, including
$G$, depends on the mathematical model underpining a specific entity
resolution task implementation.
Because the input domain of the quality metrics changes according to the
used mathematical model, the set of quality metrics available for describing
the performance of a certain entity resolution task $Q = \{q_y \mid y \in
\mathbb{N}\}$ is dependent on the mathematical model used for implementing
that task.

One possible answer to the question of whether it is possible to evaluate
an entity resolution using a theoretical model which wasn't available at the
time when the entity resolution was designed has to do with the
applicability of quality metrics.

A simple logical inference tells us that if two theoretical models represent
entity profiles in the same way, they also share quality metrics.
Essentially, if the entity resolution $e_1$ can be evaluated using a set of
quality metrics $Q_1$ and $e_2$ can be evaluated using a set of quality
metrics $Q_2$ such that $Q_1 \cap Q_2 \neq \emptyset$, we have a set of
quality metrics $Q_\cap = Q_1 \cap Q_2$ that allows us to compare $e_1$ and
$e_2$ based on $Q_\cap$.

    \subsection[models]{Entity Resolution Models}\label{sec:models}

    The representation of entity resolution results depends upon the theoretical
    model we use.
    There are multiple theoretical models available for entity resolution and
    they range from ones based on complex networks and graph theory\cite{Li2020}
    to ones based on probabilities\cite{fs1969} or
    algebra\cite{Tal11,Ben2009Swoosh}.

    John R. Talburt provides an overview of these models in his book ``Entity
    Resolution and Information Quality''\cite{Tal11}.
    The main models presented there are the Fellegi-Sunter~\cite{fs1969} model,
    the Stanford Entity Resolution Framework~\cite{Ben2009Swoosh} model and the
    algebraic\cite{tal2007algebraic} model.
    This paper aims to analyse two of the models described in that book with an
    aim to understand how they influence our ability to measure the quality of
    entity resolution.

    The Fellegi-Sunter model is the best known model for entity resolution.
    Most of quality evaluations in the entity resolution space are made using
    probabilistic metrics defined by this model.
    When measuring results using these metrics, a common experience is not being
    able to fully explain how the quality of the results output by a particular
    entity resolution process varies with the data sets the process operates on.
    A common attitude towards this lack of understanding is to chalk things off
    to the characteristics of the underlying data and the size of the dataset.
    This paper attempts to find out if there aren't some things we might be able
    to infer regardless of those two qualities of the underlying data.

    The second model this paper looks into is the elegant algebraic model.
    The metrics we most strongly associate with this model should be very
    familiar to data clustering afficionados.
    It seems to be a good complement for the Fellegi-Sunter model.

    Additionally to presenting the two models, the question of whether they can
    be used in conjunction to analyse the same result needs an answer.
    The paper describes how an observer can switch from a probabilistic
    perspective on entity resolution to an algebraic one.

    The complexity of the discussion and the presentational space required,
    force us to leave the Stanford Entity Resolution Framework model out of the
    conversation for the time being.
    This means that we won't go into the metrics that are more commonly linked
    to it either, such as the generalized merged distance, the basic merge
    distance or variation of information~\cite{Men10}.

    \subsubsection[fsm]{Fellegi-Sunter Model}\label{subsec:fsm}
    In the late 1960s Ivan Fellegi and Alan Sunter wrote the seminal
paper\cite{fs1969} for what was called record linkage and would later become
known as entity resolution.
To this day, the probabilistic model for entity resolution is this problem's
most popular formalization.
In this mathematical model, entity resolution is a function that aids in 
probabilistic decision making.

\subsubsection{Description}\label{subsubsec:F-S Model Description}

The original Fellegi-Sunter (F-S) model defines record linkage as an operation over two input
sets, $A$ and $B$.
The result of the operation is a set containing pairs of items
$X = \{(a, b), a \in A, b \in B\}$ or $X = A \times B$.

The model then deconstructs $X$ to two disjoint subsets $X = M \cup U$:
\begin{itemize}
    \item $M$ for pairs that contain matching items, and
    \item $U$ for pairs that contain non-matching items,
\end{itemize}
or more formally:
\begin{align}
    M &= \{(a, b) | a == b, a \in A, b \in B\}~\textrm{and}\nonumber \\
    U &= \{(a, b) | a \neq b, a \in A, b \in B\}\nonumber
\end{align}

The model proposes that if $a \in A$ and $b \in B$ are two vectors, their
comparison will also be a vector denoted $\gamma \in \varGamma$, where
$\varGamma$ denotes the set of all possible values of $\gamma$.
The comparison between $a$ and $b$ is performed between each corresponding
element in each of the two vectors.
The elements of $\gamma$ vary according to the type of comparison that is
performed\cite{winkler1990}.

In this context, a match decision function can make one of three decisions
regarding a pair from $X$ according to the F-S model:

\begin{itemize}
    \item consider it a \textit{link ($A_1$)} between the items;
    \item consider it a \textit{non-link ($A_3$)} between the items, and
    \item leave things undecided, thus marking it as a \textit{possible link
          ($A_2$)}.
\end{itemize}

Then a linkage rule is a function $L:\varGamma \rightarrow D$,
$D=\{d(\gamma)\}$, where

\begin{align}
    d(\gamma) = &\{P(A_1|\gamma),P(A_2|\gamma),P(A_3|\gamma)\}\textrm{, so
    that:}\nonumber\\
    &\sum_{i=1}^{3}P(A_i|\gamma) = 1\nonumber, \gamma \in \varGamma\nonumber
\end{align}

In other words, given a $\gamma \in \varGamma$, we have the linkage rule
\[
    L(\gamma) = \{P(A_1|\gamma), P(A_2|\gamma), P(A_3|\gamma)\}
\]
so that the probabilities sum up to $1$.

The F-S theorem defines it as the linkage rule that minimizes
$P(A_2|\gamma)$.
In other words, the optimal linkage rule according to the F-S model is the
linkage rule without uncertainties.

\begin{defn}
    The optimal linkage rule according to the F-S theorem is the entity
    resolution for two given input populations.
\end{defn}


This inclination towards optimization makes the model well suited for both
rule-based\cite{oyster2012} and machine learning\cite{deepm2020}
implementations.

The model accentuates two conditional probabilities:

\begin{align}
    m(\gamma)&=P(\gamma(a, b) | (a, b) \in M)~\textrm{and}\nonumber\\
    u(\gamma)&=P(\gamma(a, b) | (a, b) \in U)\textrm{.}\nonumber
\end{align}

\noindent
$m(\gamma)$ is the probability of a $\gamma$ comparison vector given an
$(a, b)$ \textit{link}.
$u(\gamma)$ is the probability of a $\gamma$ comparison vector given an
$(a, b)$ \textit{non-link}.
For brevity, we have not used the complete notation pertaining to records
from the original paper.

The two probabilities outlined above are important because they help us
express the probabilities of the Type I and Type II statistical errors
associated with the decision function that sits at the core of the linkage
rule definition above.
The probabilities of the two types of error are expressed as:

\begin{align}
    \mu&=\sum_{\gamma \in \varGamma}u(\gamma)P(A_1|\gamma)\textrm{,~and}\nonumber\\
    \lambda&=\sum_{\gamma \in \varGamma}m(\gamma)P(A_3|\gamma)\nonumber
\end{align}

where:

\begin{itemize}
    \item $\mu$ represents the Type I error of items that were erroneusly
    linked (i.e pairs in $M$ that do not belong in $M$), and
    \item $\lambda$ represents the Type II error of items that were
    erroneously \textit{not} linked (i.e~pairs in $U$ that do not belong in
    $U$).
\end{itemize}

While the model shows that the classification of pairs in link, non-link and
probable link is optimal in terms of statistical power, sound criticism of the
model was made in literature~\cite{tancredi2011fsmcrit}.
The criticism centers mostly around the assumptions of probabilistic
independence between various events described in the model~\cite{winkler2014matching,tancredi2011fsmcrit}.
While the criticism is constructive and leads to substantial improvements added
to the original model, the substrate of the improved models remains a
probabilistic one.
The quality metrics employed to determine the fitness of an entity resolution
result remain the same as those granted by the original model.

In addition to improving the model's fundamentals, work was done to address some
known limitations, such as the constraint of the model which limits the amount
of input populations to two~\cite{sad2013genfsm,Kon19}.
The assumption that the input data sources don't contain duplicates within
them~\cite{fs1969,sad2014fsmdup} only compounds risk that this assumption does
not hold true the more input data sources the model allows.
Then there is the issue of entity resolution transitivity: if data \textit{A}
and \textit{B} refer to the same entity, and data \textit{A} and \textit{C}
refer to the same entity then they all refer to the same entity~\cite{Tal11}.
For these two reasons and to be able to focus on the essential tools provided to
us by the probabilistic model of entity resolution, we will not pursue the
derived probabilistic models that can handle more than two input data sources. 

\subsubsection{Terminology Mapping}\label{F-S Terminology Mapping}

To link the Fellegi-Sunter model to our own terminology, let's start by talking
about `$a \in A$' and `$b \in B$'.
The F-S model calls $A$ and $B$ ``populations'' and $a$ and $b$,
``population elements''.
A distinction is made between records, denoted with $\alpha(a)$ and
$\beta(b)$, and population elements.
Most of the original paper refers to records.
Records seem to reflect what the computer knows during the entity resolution
process, whereas $a$ and $b$ seem to be things a human would know and would be
present in the input data sources.
In short, population elements describe \textit{entities} and records are
\textit{entity references} in our terminology.
We can think of the mappings $\alpha(a)$ and $\beta(b)$ as \textit{traits} in
our terminology.
The $A$ and $B$ populations equate to data sources.
A ``population'' is a particular type of data source because it is defined
as a set.

The Fellegi-Sunter model works under the assumption that there are two
populations $A$ and $B$ and defines the linkage rule as a function that uses
conditional probabilities that involve both populations.
On the other hand, our $Ref$ domain is solitary.
Using our terminology we would denote $A$ with $S_1$ and $B$ with $S_2$.
We start constructing our $Ref$ domain by using traits to extract $n_1$
entity references from $S_1$ and $n_2$ entity references from $S_2$.
Then $Ref = \{r_{{S_i}{k_j}},~1 \leq k_j \leq n_i, i \in {1, 2} \}$.
If, say there is 5 elements in $S_1$ and 10 elements in $S_2$ then $k_1$ goes
from 1 to 5 and $k_2$ goes from 1 to 10.

In this perspective, the extraction process outputs the adnotated set equivalent
to $Ref$, where the adnotations mark the original source (in the sense of $S_i$)
of the entity reference.

The result domain is expressed as $X = A \times B$ in this model.
It is subsequently split into $M$ and $U$ the sets of matching and
non-matching pairs in $X$, respectively.

The $X$ in the original model is equivalent to $Res$ in our terminology.
Since $Res$ is always a collection of entity profiles according to
Section~\ref{sec:Entity Resolution Formalization}, we must simply formalize the definition
of the entity profile domain \textit{P} as a set of pairs of entity references:
\begin{align}
    P = &\{(r_{{S_i}{k}}, r_{{S_j}{l}})\mid~1 \leq k \leq |S_i|,
    1 \leq l \leq |S_j|\}\textrm{, so that}\nonumber\\
    &i \neq j, i,j \in \{1, 2\}\textrm{,}\nonumber
\end{align}
where $|S_i|$ denotes the number of entity references extracted from the
source $S_i$.

With this specification, the $Res$ domain is comprised of sets of entity
profiles that are the result of performing the matching and clustering steps of
the entity resolution process.
The two steps are performed simultaneously because clustering is the same as
matching in this model.
That leads to representing entity profiles as pairs of entity references under
the original F-S model.

For example, let's say $r_{11}=(1, 2, 3)$ and $r_{21}=(4, 5, 6)$ are two entity
references that refer to the same real-world entity.
Under this model an entity profile would be represented as the following pair:
$((1, 2, 3), (4, 5, 6))$.
$M$ is the set containing the entity profiles that refer to entities, while
$U$ is the set of entity profiles that don't refer entities.
Consequently, according to this model, M is the result of entity resolution.

A natural question is whether we should also include $U$ in the result.
Since $M$ will grow proportionally to the maximum number of entity references
extracted from a single data source and $U$ grows proportionally to
$A \times B$, we propose constructing $M$ and inferring $U$ based on $M$ for
practical reasons concerning memory space and processing time.

Another concern is whether the entity resolution result is uniquely determined
for a given input under this model.
There is evidence in literature that solutions under this model converge towards
a unique solution when given the same input data\cite{winkler2014matching}.

With the above definition for $Res$ and taking into account the limitations
expressed in the previous paragraph~(\ref{subsubsec:F-S Model Description}),
we observe that this model explains matching well.
Clustering entities only two at a time looks like a form of information loss.
As stated before, not being able to describe the result in terms of how well the
entity resolution task discovered transitive links between entity references
seems like a shortcoming.

\subsubsection{Specific Metrics}\label{subsubsec:F-S Quality Metrics}

Because we have a clear definition of the Type I and Type II errors, we also
have clear definitions of true and false positives and true and false negatives.
This allows us to use any of the familiar statistical metrics that rely on
these concepts.

In defining true/false positives/negatives we use the sets $M$ and $U$ as
they are defined in the original paper describing the F-S model.
Depending on where a pair (output by the entity resolution function) is
expected to be found, we define:

\begin{itemize}
    \item \textbf{true positives} as pairs predicted to be in $M$ that
    should be in $M$,
    \item \textbf{false positives}, or type I errors, as pairs predicted to
    be in $M$, but should be in $U$,
    \item \textbf{true negatives} as pairs predicted to be in $U$ that
    should be in $U$, and
    \item \textbf{false negatives}, or type II errors, as pairs predicted to
    be in $U$, but should be in $M$.
\end{itemize}

Quality metrics for this model compare a ground truth, consisting of ideal
tuple pairs, with the entity resolution task's output.
$M$ is the ground truth in this context.
Due to the impracticality of using $U$, our focus will be on concepts
related to $M$: true positives, false positives, and false negatives.

Several metrics based on these concepts exist, though the effectiveness of
some has been questioned\cite{Goga2015}.
With this in mind we choose three quality metrics that are widely used with
this model:

\begin{align}
Precision &= \frac{TP}{TP+FP}\nonumber \\
Recall &= \frac{TP}{TP+FN}\nonumber \\
F_1 &= 2 \cdot \frac{Precision \cdot Recall}{Precision+Recall}\nonumber
\end{align}

These metrics fit our quality metric definition from~\ref{sec:Entity Resolution Formalization}
and are defined on $Res^2$ with their result falling within the
$\left[0, 1\right] \subset \mathbb{R}$ closed interval.
Their input consists of two entity resolution results (represented as presented
previously) and their output is always a real number.

\textit{Precision} (or the positive predictive value) is defined as the
number of correct predictions that were made in relation to the total number
of predictions that were made.

\textit{Recall} (or sensitivity) is defined as the number of correct
predictions that were made in relation to the total number of positive
predictions that could have been made (which corresponds to the number of
items in the ground truth).

The \textit{$F_1$} score is the harmonic mean of the precision and the
recall and it is used to capture the tradeoff between precision and
recall\cite{hitesh2012}.



    \subsubsection[algebraic]{Algebraic Model}\label{subsec:algebraic}
    
Initially proposed as a model to discern information quality in entity
resolution over large datasets\cite{tal2007algebraic}, the algebraic model
for entity resolution was later refined to describe entity resolution
itself\cite{Tal11}.
While the Fellegi-Sunter model described in Subsection~\ref{subsec:fsm} is a
probabilistic model that partitions input in two data sources to compare,
this model allows as many information sources as necessary.
Another improvement is the support for transitive matching.

\subsubsection[algdesc]{Description}\label{subsubsec:algdesc}

The core tenet of this model is that entity resolution is an \textit{
equivalence relation}, in the sense defined in set theory.

\begin{defn} If $X$ is a set and $R$ is a rule that takes two elements from
$X$, $x$ and $y$, and tells us whether $x$ is in relationship $y$ in the way
denoted with $R$, then we call $R$ a \textit{binary relation} on
$X$\cite{hoffman1971linear}.
\end{defn}

We can represent $R$ as a function $R:X \times X \rightarrow \{0,1\}$ where
$R(x,y)=1$ if $x$ is in relation $R$ with $y$ and $R(x,y)=0$ otherwise.

\begin{defn}If $R$ is a binary relation on the set $X$, it is convenient to
write $xRy$ when $R(x, y) = 1$.
A binary relation R is called:

\begin{enumerate}
    \item reflexive, if $xRx, \forall x \in X$;
    \item symmetric, if $yRx \implies xRy$, $\forall x,y \in X$;
    \item transitive, if $xRz \land yRz \implies xRz$, $\forall x,y,z \in X$.
\end{enumerate}

An \textit{equivalence relation} on $X$ is a reflexive, symmetric, and
transitive binary relation on $X$\cite{hoffman1971linear}.
\end{defn}

Note that equivalence relations seem to describe almost perfectly the
matching step in entity resolution.
Let's also observe that equivalence relations work on sets for now and move
on.

An equivalence relation over a set $X$ generates an equivalence class.

\begin{defn}Suppose $R$ is an equivalence relation on the set $X$.
If $x$ is an element of $X$, let $\bar{x}$ denote the set
$\{y \in X | xRy\}$.
$\bar{x}$ is called the \textit{equivalence class} of $x$ (for the
equivalence relation $R$)\cite{hoffman1971linear}.
\end{defn}

Since $R$ is an equivalence relation, the equivalence classes for $R$ have the
following properties\cite{hoffman1971linear}:
\begin{enumerate}
    \item $xRx \in \bar{x} \implies \bar{x} \neq \emptyset$,
    \item $y\in\bar{x}\iff~x\in\bar{y}$,$~\forall x,y \in X$ because
    equivalence relations are symmetric,
    \item $\forall x,y \in X$, $\bar{x}$ is either identical with $\bar{y}$
    or they have no elements in common\cite{hoffman1971linear,Tal11}.
\end{enumerate}

The last fundamental notion from set theory that is needed to construct the
algebraic model of entity resolution is the \textit{partition} of a set.

\begin{defn}
    A \textit{partition} of a set $X$ is a collection of non-empty disjoint
    subsets of $X$ whose union is $X$\cite{halmos1960naive,Tal11}.
\end{defn}

Things come together starting with the simple observation that an element
belonging to one of the sets from a partition over $X$ is in a relation with
itself and other elements from the same set.
We say that this relation is \textit{induced} by the partition.
Conversely, from the point of view of the above relation, we say that it
\textit{generated} the partition.

When the partition in question is the set of equivalence classes over $X$,
then the induced relation is an equivalence relation.
Conversely, an equivalence relation $R$ in $X$ will generate a partition
over $X$ comprised by the equivalence classes of $X$\cite{halmos1960naive,
Tal11}.

\begin{defn}
    The equivalence relationship over a set of entity references that exists
    when two references from that set refer to the same real-world entity is
    called \textit{entity resolution}.
\end{defn}

Let's denote this relation with $ER$ and state that if $xERy$ then x and y
are \textit{linked}\cite{Tal11}.
Taking our two earlier mental notes, we can now say that applying $ER$ to
individual elements of a set is akin to the \textit{matching} step in the
entity resolution process.
With each application, we now know that $ER$ creates a partition over the
initial set.
This side-effect of equivalence relationships is the \textit{clustering}
step or aspect of the entity resolution process.
The equivalence classes from the partition generated by $ER$ are also known
as \textit{clusters} in the context of entity resolution.

A last point of notice is that entity resolution tasks do not necessarily
generate a unique outcome\cite{Tal11}.
Depending on the order in which entity references are processed, runing the
same entity resolution task might generate different partitions.
The entity resolution tasks where the order in which entity references are
processed is irrelevant are called \textit{sequence neutral}\cite{Tal11}.
In this paper we are only interested in sequence neutral entity resolution
tasks.

\subsubsection[algrel]{Terminology Mapping}\label{subsubsec:algrel}

In writing this paper, we used a lot of notions that are specific to the
algebraic model.
For example, attributes and entity references preserve their meaning.
In fact, it can be safely assumed that all common terms preserve their
meaning.

The limitations of the algebraic model stem from its need to deal with input
data that is organized as a set.
That means that input data sources that contain duplicate data need to be
cleaned during the extraction step of the entity resolution process before they
are usable in the matching and clustering steps.

The result of an entity resolution task is a partition over the input set of
entity references.
We can then define $P$, the domain of entity profiles as a partition over the
input set of entity references.

\begin{align}
    P &= \{C_i\mid~C_i\subset~Ref,~i\in\mathbb{N}\}\nonumber\\
    Res &= \{C_1,~C_2,~C_3,~\ldots,~C_n\mid~C_i\in~P\}\textrm{, so that}\nonumber\\
    C_i&\neq\emptyset,\forall~1\leq~i\leq~n\textrm{,~and}\nonumber\\
    C_i\cap~C_j&=\emptyset,\forall~1\leq~i\neq~j\leq~n\textrm{,~and}\nonumber\\
    \bigcup_{i=1}^{n}C_i&=Res\nonumber
\end{align}

$Res$, the output domain of the entity resolution is the domain of all
partitions over $Ref$.
Therefore, any item in $Res$ is a collection of profiles, as defined in
Section~\ref{sec:entity-resolution}.
In the algebraic model the profiles that are grouped together in a result also
constitute a partition over $Ref$.
This implies that any entity reference in $Ref$ will be a part of one and only
one entity profile that is part of an entity resolution result.

\subsubsection[algeval]{Specific Metrics}\label{subsubsec:algeval}

Building on top of the notions we already know about ground truth and entity
resolution results, we denote the ground truth with $G$.
The partition that is generated by the entity resolution under scrutiny will be
denoted with $R_{ER}$. 
The quality metrics for entity resolution under the algebraic model have to
do with computing the similarity of partitions~\cite{hitesh2012}.
The ones that we chose to cover seemed to be both the more fundamental and the
more widely used among these metrics.

The first among these is the Rand index introduced by William Rand in
1971\cite{rand1971}.
This index computes the similarity of partitions by comparing pairs of
elements formed from the elements of the clusters of each partition.
Supposing that:
\begin{itemize}
    \item $a$ is the number of pairs that are shared between clusters of $G$
    and $R_{ER}$,
    \item $b$ is the number of pairs that are only in clusters in $G$, but
    not in clusters in $R_{ER}$,
    \item $c$ is the number of pairs that are only in clusters in $R_{ER}$, but
    not in clusters in $G$,
    \item $d$ is the total number of pairs that are not shared.
\end{itemize}
then the Rand index is defined by the quantity
$\frac{a+d}{a+b+c+d}$\cite{adjrand2001}.

The Rand index, $RI : Res^2 \rightarrow \left[0, 1\right]$ is a quality metric
in the sense defined in Section~\ref{sec:entity-resolution}.
The higher the score, the better.
It is usually used to rate the similarity of two partitions.
$1 - Rand Index$ can be an effective measure for the distance between two
partitions.

The expected value of the Rand index of two random partitions does not take
a constant value (say zero)~\cite{adjrand2001}.
That leads to the introduction of the adjusted Rand index which follows the
same principles of comparing partitions in a pairwise manner, but uses a
specific distribution for randomness~\cite{adjrand1985}.
Without going into detail, we can provide a formula for the adjusted Rand
index by using the quantities we defined for the Rand index~\cite{Tal11}.
\[
    Adjusted~Rand~Index = \frac{
        a -\left(\frac{(a+b)\cdot(a+c)}{a+b+c+d}\right)
    }{
        \frac{(2\cdot a+b+c)}{2}-\left(\frac{(a+b)\cdot(a+c)}{a+b+c+d}\right)
    }
\]

Higher values of this index mean more similar partitions.
The adjusted Rand Index, $ARI : Res^2 \rightarrow \left[0, 1\right]$, is a
quality metric in the sense defined in Section~\ref{sec:entity-resolution}.
It is bounded above by 1 and takes on the value 0 when the index equals its
expected value, with values lower than zero having no substantive
value~\cite{adjrand1985}.

The third quality metric we can use for comparing partition similarity is
the Talburt-Wang index (TWI).
The merit of this metric is that it closely follows the Rand index, but is
much simpler to compute~\cite{Tal11}.
\[
    TWI = \frac{\sqrt{|G|\cdot|R_{ER}|}}{|G \cap R_{ER}|},
\]
where if $X$ is a set, then $|X|$ represents the numer of elements in $X$. 

The $TWI: Res^2 \rightarrow \left[0, 1\right]$ is a quality metric as defined in
Section~\ref{sec:entity-resolution}.
It uses the number of clusters that are found in both the ground truth and in
the entity resolution task's output.
It provides a good approximation of how many clusters are returned correctly
by the entity resolution task.
The Talburt-Wang index takes values between 0 and 1, with 1 indicating that
the return value of the entity resolution task perfectly matches the ground
truth.

Other similarity metrics used to compare partitions are the so-called pairwise
and cluster metrics.
Pairwise metrics were first introduced as alternative quality metrics to be used
for evaluating entity resolution results\cite{Men10}.
% \textcolor{green}{eu as include si referintele "radacina": https://www.jstor.org/stable/pdf/1756667.pdf, https://www.cs.utexas.edu/users/ml/papers/marlin-kdd-wkshp-03.pdf}
% \textcolor{orange}{as vrea sa le includ, dar nu sunt sigur ca le pot include aici. referintele nu par sa aiba o legatura directa cu pairwise metrics. poate in alta parte?}

Similarly to the Rand Index, pairwise metrics rely on extracting pairs of items
to determine how similar two partitions over the same set are.
They differ from the Rand index because pairwise metrics operate at the more
granular partition level and not at the partition class level.

Pairwise precision, $PP:Res^2\rightarrow~\left[0,1\right]$ is defined as the
ratio between the number of pairs in the  that are also in $G$ and the number
of pairs in $R_{ER}$.
In other words pairwise precision tells us how many pairs the entity
resolution task got right out of the pairs it returned with a maximum value of
1 indicating an ideal result.
\[
    PP = \frac{|Pairs(R_{ER}) \cap Pairs(G)|}{|Pairs(R_{ER})|}
\]

Pairwise recall, $PR:Res^2\rightarrow~\left[0,1\right]$ is defined as the ratio
between the number of pairs in $R_{ER}$ that are also in $G$ and the number of
pairs in $G$.
In other words, pairwise recall tells us how many pairs the entity
resolution task got right from the ideal answer, with a maximum value of 1
indicating an ideal result.
\[
    PR = \frac{|Pairs(R_{ER}) \cap Pairs(G)|}{|Pairs(G)|}
\]

The harmonic mean between pairwise precision and pairwise recall,
$PR:Res^2\rightarrow~\left[0,1\right]$ is called the pairwise F1 score or the
pairwise comparison measure\cite{Men10}.
\[
    PF1=\frac{2 \cdot PP \cdot PR}{PP + PR}
\]
The maximum value of this measure is 1 and indicates ideal precision and ideal
recall.

Similarly to their pairwise counterparts, we also have cluster metrics: cluster
precision, cluster recall and cluster comparison measure (or $CF_1$).

The cluster precision, $CP:Res^2\rightarrow~\left[0,1\right]$ is the ratio
between the number of identical clusters between the entity resolution output
and the ground truth and the number of clusters in the entity resolution output.
In other words, we want to know how many clusters the entity resolution task
got right in relation to its own output.
The maximum value of 1 indicates perfect coverage of the output in relation to
the ground truth.
\[
    CP=\frac{|Clusters(R_{ER}) \cap Clusters(G)|}{|Clusters(R_{ER})|}
\]

The cluster recall $CR:Res^2\rightarrow~\left[0,1\right]$ is the ratio between
the number of clusters that are common between the entity resolution task's
result and the ground truth over the number of clusters in the ground truth.
In other words, this measures how many clusters did the entity resolution
task get right in relation to the ground truth.
The maximum value of 1 indicates full coverage of the ground truth by the
output.

\[
    CR = \frac{|Clusters(R_{ER}) \cap Clusters(G)|}{|Clusters(G)|}
\]

Finally, the cluster comparison metric $CF_1:Res^2\rightarrow~\left[0,1\right]$
is the harmonic mean between the cluster precision and the cluster recall.

\[
    CF_1 = \frac{2 \cdot CP \cdot CR}{CP + CR}
\]
The maximum value indicates perfect cluster precision and cluster recall, so a
perfect match.

The pairwise and cluster metrics provide more insight into how two entity
resolution results compare structurally, which is an ability that statistical
metrics lack.
The idea of discerning how the structures of two results compare was first
suggested in Rand's paper that described his namesake index\cite{rand1971}.

The pairwise quality metrics (the pairwise precision, recall and
comparison measure as well as the Rand Index and Adjusted Rand Index) mimic
more or less closely the statistical metrics of the F-S model.
The cluster quality metrics exposed by this model provide insights that
either complement or support the pairwise metrics.


    \subsubsection{From Probabilities to Algebra}\label{subsec:fsm-alg}
    
\textcolor{green}{sub-sectiunea aceasta e f incitanta si cred ca reprezinta una din contributiile importante ale lucrarii; nu mi-e clar daca se poate prezenta ca o "aliniere" originala a celor 2 modele (adica daca s-a mai facut asa ceva in literatura pana acum in acelasi stil sau , daca s-a facut, dar s-a facut in alt fel) sau sub forma unor discutii/analize f revelatoare}
    
As stated before, one of the misgivings of the Fellegi-Sunter model is that
it captures relatively well the matching aspect of entity resolution without
doing as good a job for its clustering aspect.
The algebraic model allows a more complete perspective over how entity
resolution works by supporting measures of both matching (pairwise metrics)
and clustering.
One of the questions we asked initially is whether we could transition from
one model to the other to enjoy the more complete overview offered by the
algebraic model.

The answer to this question lies in the difference between how the two
models represent entity profiles and entity resolution output in general.
Intuitively, this becomes a question of whether we can automatically convert
sequences of pairs of items to clusters.
Then, if that is possible, the question becomes whether from a set of pairs
of items from an input set we can obtain a partition over the input set
itself.

The idea is to consider pairs of items as being edges in a graph.
Then we can take inspiration from Kruskal's algorithm to construct connected
the components of this graph.
We do this by utilizing the Union-Find data structure 
\textcolor{green}{add a reference!}
.
This very well known data structure that is used for constructing sets of
data incrementally can be used to construct the graph components as sets.

The algorithm takes as input a set of items and an iterable sequence of
pairs of items from that set.
The Union-Find data structure is initialized with all items in the set.
Then, with each pair it processes, it links items that are paired up in
the same `class'.

We can see that all of the items in the set are used and none are discarded.
The Union-Find data structure produces stable classes of items.
In case an item is part of two classes, these classes are merged.
As a result, all sets produced by the Union-Find data structure are
distinct.
Therefore the Union-Find data structure produces a partition over the input
set of items.

This algorithm can be applied to meet our needs by leveraging the input domain
$Ref$ and the representation of the entity resolution result in the F-S model.
Recall that $Ref$ is always defined as a set of entity references.
We defined the output of the F-S model as a set of pairs of entity
references.
The ground truth expressed according to the F-S model and the $Ref$ set of
entity references are the input parameters of the Union-Find data structure.

The ground truth always contains the ideal entity resolution output
regardless of the mathematical model.
It cannot be incomplete or faulty, by definition.
Therefore, when we apply the Union-Find data structure to the ground truth
it cannot return anything less than a ground truth because it processes all
the data in $Ref$ and the partition it creates over $Ref$ is stable (i.e
given the input sequence of pairs it cannot produce two partitions over $Ref$
on two separate occasions).

If we have a ground truth expressed as an iterable sequence of matching
pairs and we know the input $Ref$ domain that is used for entity resolution,
we obtain an ideal partition induced by entity resolution over $Ref$.

        
    \section{Experiments}\label{sec:experiments}
    So far we have shown how entity resolution models influence data structures
that are part of the application programming interface of an entity resolution
computer algorithm.
In the introduction we were asking whether there might be invariant conditions
that are not dependent on the data on which we perform entity resolution.
The experiments we have designed are meant to find some invariant conditions
that depend on a particular entity resolution model.

Secondly, we want to determine which metrics (if any) are more reliable in the
face of changes in underlying data.
To do that, we use the same entity resolution algorithm across multiple data
sets and manipulate its configuration in ways that would result in predictable
outcomes.

Third, we have stated an interest in switching from one entity resolution model
to another to gather more information using the same ground truth.
In our experiment we want show both the similarity and the complementarity of
some of the metrics from an entity resolution model with respect to metrics from
another entity resolution model.
This train of thought leads to a comparative analysis of the metrics we obtain
by running one entity resolution algorithm on different data sets.

The code for running the experiments is available on GitHub~\cite{matchescu}.
Running the experiment does not require any specific hardware.
We use the Python~\cite{python} programming language (version 3.11) and the
Numpy~\cite{numpy} and Pandas~\cite{pandas2023} libraries for efficient
data manipulation.
All metrics are computed using an OpenSource library available on GitHub as
well~\cite{matchescu-er-metrics2023}.

\subsection{Datasets}\label{subsec:Experiment Datasets}


To ensure we are working with standard data that simulates real data and which
is widely used and therefore familiar, we use the Abt-Buy, DBLP-ACM and
Amazon-GoogleProducts~\cite{vldb2010} benchmark data sets.
We did not use the DBLP-Scholar data set from the same source because of
practical reasons concerning the run time of the experiment on the available
hardware.

For our initial small scale experiments we built our own control
dataset~\cite{expdata2023} from the `Buy' component of the `Abt-Buy' data set.

To ease experimentation, we use the original tabular representation
which allows us to take advantage of Pandas~\cite{pandas2010,pandas2023}
and also has the benefit of not requiring adaptation to an internal format.
The tabular data is stored on disk as CSV files.
Each CSV file has a first row containing the column headings and subsequent
rows containing records within which the fields are separated by comma.
The data representation, however, is not important for the experiment's
outcome. All experiment datasets are available on GitHub\cite{expdata2023}.

\begin{table*}[htbp]
    \centering
    \begin{tabular}{lllccc}
        \toprule
        \multicolumn{3}{c}{\textbf{General Information}} & \multicolumn{2}{c}{\textbf{Size}} & \multirow{2}{*}{\textbf{Ground Truth Size}}\\
        \cline{0-4}
        \textit{Domain} & \textit{Mapping Columns} & \textit{Sources} & \textit{Source 1} & \textit{Source 2} & \\
        \toprule
        \multirow{4}{*}{Bibliographic} & \tabitem~title & \multirow{4}{*}{DBLP-ACM} & \multirow{4}{*}{DBLP=2616} & \multirow{4}{*}{ACM=2294} & \multirow{4}{*}{2224} \\
        & \tabitem~authors & & & &\\
        & \tabitem~venue & & & &\\
        & \tabitem~year & & & &\\
        \midrule
        \multirow{4}{*}{E-commerce} & \tabitem~product name & \multirow{2}{*}{Abt-Buy} & \multirow{2}{*}{Abt=1081} & \multirow{2}{*}{Buy=1092} & \multirow{2}{*}{1097} \\
        & \tabitem~description & & & &\\
        \cline{3-6}
        & \tabitem~manufacturer & \multirow{2}{*}{Amazon-GoogleProducts} & \multirow{2}{*}{Amazon=1363} & \multirow{2}{*}{GoogleProducts=3226} & \multirow{2}{*}{1300} \\
        & \tabitem~price & & & &\\
        \midrule
        \multirow{4}{*}{Control} & \tabitem~product name & \multirow{4}{*}{mini-Buy} & \multirow{4}{*}{DG1=10} & \multirow{4}{*}{DG2=10} & \multirow{4}{*}{10} \\
        & \tabitem~description & & & &\\
        & \tabitem~manufacturer & & & &\\
        & \tabitem~price & & & &\\
        \bottomrule
    \end{tabular}
    \caption{Experiment Data Characteristics}\label{tab:dsattrs}
\end{table*}

The characteristics of the experiment data sets are available in
Table~\ref{tab:dsattrs}. Additional information concerning the benchmark
data sets can be found in the original paper~\cite{vldb2010}.

We use the benchmark datasets almost in their original forms.
In order to make entity matching algorithms that use prefix matching, we make a
slight adjustment to the benchmark datasets and move the `id' column to the last
position in each data set.
We also use the ideal mapping provided with each data set to establish the
ground truth in the probabilistic model representation.

With our tiny custom dataset, the goal is to exert complete control over the entity
resolution process.
This is achieved by creating custom input data sources for the entity resolution
process through data generation.
The data sources each contain ten chosen items from the `Buy' subset of the
`Abt-Buy' dataset.
The chosen items exacerbate faults in data.
There are many empty or missing attribute values and there are many
near-duplicate entity references.
Our intention was to have an unrealistically bad data set.

The data generation procedure divides the ten-item subset into two distinct
tables:

\begin{enumerate}[label=\textbullet,leftmargin=1cm]
\item DG1, with columns: \texttt{`name', `manufacturer', `price', `id'}
\item DG2, with columns: \texttt{`description', `name', `id'}
\end{enumerate}

These tables serve as the foundational input for our experimental analysis.
The construction of the ground truth, as a list of paired elements, is
straightforward due to this data generation approach.
We operate under the assumption that each item in our `Buy' subset
corresponds uniquely to a single real-world entity.
Consequently, the ground truth consists of pairs, each pair representing a
distinct real-world item.

Tables~\ref{tab:buy-record},~\ref{tab:dg1-record} and~\ref{tab:dg2-record} contain example
records indicative of our data generation procedure.

\begin{table}[ht]
    \setlength\tabcolsep{6pt}
    \centering
    \begin{tabular}{lllll}
        \toprule
        id&name&description&manufacturer&price\\
        \midrule
        205554724&Seiko SXDA04& & &\$138.00\\
        \bottomrule
    \end{tabular}
    \caption{Example `Buy' record}\label{tab:buy-record}
\end{table}

\begin{table}[ht]
    \setlength\tabcolsep{12pt}
    \centering
    \begin{tabular}{llll}
        \toprule
        name & manufacturer & price & id \\
        \midrule
        Seiko SXDA04 & & \$138.00 & 205554724 \\
        \bottomrule
    \end{tabular}
    \caption{Example `DG1' record}\label{tab:dg1-record}
\end{table}

\begin{table}[ht]
    \setlength\tabcolsep{12pt}
    \centering
    \begin{tabular}[b]{lll}
        \toprule
        description&name&id \\
        \midrule
        &Seiko SXDA04&205554724 \\
        \bottomrule
    \end{tabular}
    \caption{Example `DG2' record}\label{tab:dg2-record}
\end{table}

Using the extraction traits described before would yield us the
following entity references from DG1 and DG2, respectively:
\begin{enumerate}
    \item \texttt{(`Seiko',`SXDA04',`\$138.00',`205554724')},
    \item \texttt{(`Seiko',`SXDA04',`205554724')}.
\end{enumerate}

We will refer to our generated dataset with the `mini-buy' moniker.

A consistent aspect with both the benchmark and the generated data sets is that
the data undergoes uniform conversion to string format, emphasizing word
extraction.
The pipeline ensures a left-to-right, top-to-bottom arrangement of extracted
words, with column and row order preservation, thus providing the entity
resolution task references consisting of a single text attribute.
This attribute amalgamates words from all attributes in the source data records
into a cohesive text unit per record.
We employ this process to effectively ensure sequence neutrality.

\subsection{Entity Resolution Algorithm}\label{subsec:Entity Resolution Algorithm}

In searching for invariant conditions that hold true regardless of the data set
that we use for experimentation, it is tempting to say that the choice of entity
resolution algorithm is non-consequential.
Given that this experiment is our first foray into the matter, it seems wise not
to give in to temptation.
There are two limitations that we find important in choosing an appropriate
entity resolution algorithm.

Firstly, the algorithm should perform reasonably well without losing the ability
to easily explain its outcome.
In this context we are thinking primarily about quantitative performance marks
such as resource consumption or time spent.

Secondly, we want to choose an algorithm that has very few configurations so
that its qualitative performance is easy to plot over a relatively large  number
of configurations.

The \texttt{ppjoin}\cite{ppjoin} entity matching algorithm meets these criteria.
PPJoin stands for Position Prefix Join.
It is an algorithm designed to find similarities between records by comparing
them using a prefix of each record determined by employing the Jaccard
index~\cite{jaccard1912,finley1996}.
The algorithm's outcomes are intuitively easy to explain.
A lower Jaccard threshold will cause the algorithm to take into account shorter
prefixes, thus increasing the chances of a match.
A higher threshold should match fewer items, but with less chance for error.
Throughout the experiment, the Jaccard threshold will be denoted with \textit{t}.

Our experiment consists of the following steps:

\begin{itemize}
    \item generate or load the data;
    \item take 100 entity matching passes through the input data with
          Jaccard thresholds ranging from 0 to and including 0.99 in steps of
          0.01, storing the results for each step;
    \item compute the statistical and algebraic similarity metrics for each
          of the stored results.
\end{itemize}

Experiments that use multiple threshold values to determine whether the data
set size influences duplicate detection performance were performed in the
past~\cite{draisbach2013choosing}.
Those experiments differ from our own in significant ways.

First, the thresholds were used for tuning the similarity function and not to
influence the entity resolution model.
While tuning the similarity function acts as an interference at the attribute
level, the Jaccard threshold is used by the \texttt{ppjoin} algorithm
to manipulate the number of attributes that participate in a comparison, too.

Second, these previous experiments were only concerned with evaluating how the
F-measure varies according to data set size.
In our experiments we attempt to find invariant conditions using multiple
quality metrics.
 
Lastly, we do not focus our experiments on input size except for one significant
way.
The conclusion of the referenced experiments is that the size of the data set
does in fact influence F-measure results independently of other factors.
We focus on finding predictors of entity resolution quality that do not depend
on data set size, but on the entity resolution model itself.
For this purpose, our three benchmark data sets are of similar size.
Their size is two orders of magnitude above the size of our control data set.
We look for the conditions that occur on the control set and in all benchmark
data sets.

The \texttt{ppjoin} algorithm treats entity resolution as a matching problem.
Consequently, Fellegi-Sunter is the natural entity resolution model for
understanding the quality of its results.
In order to evaluate the quality of the algorithm under the algebraic model, we
have to convert a list of matches (the data structure output by the \texttt{ppjoin}
algorithm) to a partition over a set.

For our own generated data we are able to generate the ground truth partition
for the algebraic model based on the input data.
In the case of the benchmark data sets, we need something different.

\subsection{Between Probabilities and Algebra}\label{subsec:Probabilities to Algebra}

\textcolor{green}{sub-sectiunea aceasta e f incitanta si cred ca reprezinta una din contributiile importante ale lucrarii; nu mi-e clar daca se poate prezenta ca o "aliniere" originala a celor 2 modele (adica daca s-a mai facut asa ceva in literatura pana acum in acelasi stil sau , daca s-a facut, dar s-a facut in alt fel) sau sub forma unor discutii/analize f revelatoare}
    
As stated before, one of the misgivings of the Fellegi-Sunter model is that
it captures relatively well the matching aspect of entity resolution without
doing as good a job for its clustering aspect.
The algebraic model allows a more complete perspective over how entity
resolution works by supporting measures of both matching (pairwise metrics)
and clustering.
One of the questions we asked initially is whether we could transition from
one model to the other to enjoy the more complete overview offered by the
algebraic model.

The answer to this question lies in the difference between how the two
models represent entity profiles and entity resolution output in general.
Intuitively, this becomes a question of whether we can automatically convert
sequences of pairs of items to clusters.
Then, if that is possible, the question becomes whether from a set of pairs
of items from an input set we can obtain a partition over the input set
itself.

The idea is to consider pairs of items as being edges in a graph.
Then we can take inspiration from Kruskal's algorithm to construct connected
the components of this graph.
We do this by utilizing the Union-Find data structure 
\textcolor{green}{add a reference!}
.
This very well known data structure that is used for constructing sets of
data incrementally can be used to construct the graph components as sets.

The algorithm takes as input a set of items and an iterable sequence of
pairs of items from that set.
The Union-Find data structure is initialized with all items in the set.
Then, with each pair it processes, it links items that are paired up in
the same `class'.

We can see that all of the items in the set are used and none are discarded.
The Union-Find data structure produces stable classes of items.
In case an item is part of two classes, these classes are merged.
As a result, all sets produced by the Union-Find data structure are
distinct.
Therefore the Union-Find data structure produces a partition over the input
set of items.

This algorithm can be applied to meet our needs by leveraging the input domain
$Ref$ and the representation of the entity resolution result in the F-S model.
Recall that $Ref$ is always defined as a set of entity references.
We defined the output of the F-S model as a set of pairs of entity
references.
The ground truth expressed according to the F-S model and the $Ref$ set of
entity references are the input parameters of the Union-Find data structure.

The ground truth always contains the ideal entity resolution output
regardless of the mathematical model.
It cannot be incomplete or faulty, by definition.
Therefore, when we apply the Union-Find data structure to the ground truth
it cannot return anything less than a ground truth because it processes all
the data in $Ref$ and the partition it creates over $Ref$ is stable (i.e
given the input sequence of pairs it cannot produce two partitions over $Ref$
on two separate occasions).

If we have a ground truth expressed as an iterable sequence of matching
pairs and we know the input $Ref$ domain that is used for entity resolution,
we obtain an ideal partition induced by entity resolution over $Ref$.


\subsection{Outcomes from Control Data}\label{subsec:Control Data}

\subsubsection{Fellegi-Sunter Model Results}\label{subsubsec:F-S Results}

The ground truth and the results are represented as lists of pairs.
The ground truth was built by iterating over both input data sources using
the same cursor and outputting pairs of records.
Note that there are duplicate CSV records that only differ on the `id' column.

Intuition tells us that for values of the Jaccard similarity threshold
\textit{t}\footnote[1]{the Jaccard similarity coefficient is used to determine
the length of the prefix which should be used to compare two entity references
to determine whether they refer to the same entity or not} that are either too
low or too high we should have lower precision.
For higher thresholds we should also have lower recall values, whereas for
lower values the recall should be higher.
Figure~\ref{fig:mini-buy-fs} shows the \texttt{ppjoin} results at various
values of the Jaccard threshold $t$.

\begin{figure}[htbp]
    \centering
    \captionsetup{justification=centering}
    \includegraphics[width=\columnwidth]{mini-buy-fsm-main}
    \caption{Statistical metrics on `mini-buy' dataset for results of F-S model}\label{fig:mini-buy-fs}
\end{figure}

We can make some observations based on Figure~\ref{fig:mini-buy-fs}.
For values of $t \geq 0.78$, we end up with fewer (lower recall), but more
accurate matches (higher precision) than compared to lower values of $t$ because
the amount of false positives decreases and the amount of false negatives
increases.

For $0.6 \leq t \le 0.78$, precision decreases because the false
positive rate increases and recall increases because the false negative rate
decreases.
This is in line with expectations: the prefix becomes shorter causing a higher
percentage of attributes to be identical.
However, the prefix length is not short enough so that some of the matching
entity references in the ground truth that are more different from one another
will also be captured by \texttt{ppjoin}.

Compared to the previous interval, for values of $0.36 \leq t \le 0.6$, both
precision and recall increase compared to the previous interval.
As the prefix drops, more and more items from the ground truth are matched.
The number of false negatives decreases while the number of false positives
does not increase as fast as the number of true positives.

When $t$ is nearing zero there is an expected increase in recall due to fewer
and fewer overall negatives.
Precision, expectedly decreases due to an ever higher false positive rate.

We need to make a small note on near-identical items in the control data set,
such as \texttt{208114672} and \texttt{208114673}.
For lower values of \textit{t}, the algorithm produces two true positives and
two false positives.
By itself, the increase in false positives intuitively lowers precision.
However, lower values of \textit{t} also broaden the comparison space which
now contains entity references with short values for the name attribute,
like \texttt{205554724}.
By having new items to compare, we also increase the amount of true positives.
This dynamic ends up increasing precision as we lower values of \textit{t} so
long as we are not operating on the full input set.

Note that while we do describe the dynamic specifically for the \texttt{ppjoin}
algorithm, the dynamic of having an increased number of true positives offset
precision in an unexpected direction is not algorithm specific at all.
Regardless of the algorithm that we use, given some circumstances specific for
that algorithm, it should be possible to obtain the same effect we see here.
This observation is interesting because most entity resolution users look at the
$F_1$ score thinking that it provides a good balance between precision and recall.
These users rarely analyse the components of the $F_1$ score.
This situation can lead to unexpected outcomes.
For instance, there might be a substantial drop in $F_1$ scores, which are
indicators of accuracy, when the input data for the entity resolution process is
changed.
The reason for this is that the precision or recall may be higher than expected
for the ideal configuration determined in the controlled scenario.

On the other hand, a balanced trade-off between precision and recall might not
be a requirement.
For instance, a core legal principle applicable in many jurisdictions across the
world states that is far more favourable to let 100 guilty people escape than to
put one innocent behind bars.
By that logic, we only care about precision.
Perfect recall while still retaining some precision might be useful during
research (e.g~finding papers on the same topic) or medical diagnosis (other
cases with the same symptoms).
In these cases, sifting through false positives might be preferable to missing
out on important information because of a false negative that is due to a
suboptimal input configuration.

Our control dataset (which is purposefully flawed) instructs us that the ideal
configuration values for our algorithm are:
\begin{itemize}
    \item $0.15 \leq t \leq 0.38$ for the best balanced outcome
    \item $t \leq 0.38$ for the most sensitive system
    \item $0.76 \leq t \leq 0.95$ for the highest precision
\end{itemize}

\subsubsection{Pairwise Metrics}\label{subsubsec:Pairwise Results}

The ground truth and the result of the ER process are represented as partitions
over an input set of data.
Because we include the IDs in both DG1 and DG2, the input data set will
contain 20 items.
The ground truth contains 10 pairs of items.

Measuring the similarity of two partitions can be accomplished in more ways than
measuring the statistical success of a matching process.
Among these metrics, the pairwise comparison of two partitions is the best
approximation of the statistical metrics\cite{Men10}.
Figure~\ref{fig:mini-alg-pairwise} shows the computed pairwise metrics for
various values of \textit{t}.

\begin{figure}[htbp]
    \centering
    \captionsetup{justification=centering}
    \includegraphics[width=\columnwidth]{mini-buy-algebraic-pairwise}
    \caption{Algebraic pairwise metrics on `mini-buy' dataset}\label{fig:mini-alg-pairwise}
\end{figure}

Because pairwise metrics do not support transitivity~\cite{Men10,hitesh2012},
they should be closely related to probabilistic metrics.
Indeed, we see a plot similar to the one in Figure~\ref{fig:mini-buy-fs} for statistical metrics.
The optimal configuration for a balanced entity resolution in terms of pairwise
equivalence of the result to the ground truth is very similar to the range we
observed based on probabilistic measures.
The pairwise precision plot's shape is almost identical, too.
All of these signs convince us of a successful conversion of the probabilistic
ground truth to an algebraic ground truth.

The only outlier we see is the pairwise recall plot
For $t \le 0.12$ pairwise recall drops significantly whereas statistical recall
stays at the maximum value.
The reason behind this variation lies in the difference between the two models
of entity resolution.
Whereas the standard recall formula only accounts for false negatives (of which
we can not have any using the generated data set), pairwise recall requires all
the result data to be partitioned and singleton clusters play an ever more
important part in the computed recall score as their number increases.

This situation leads to presenting our \textbf{first invariant condition
candidate}.
\textbf{Probabilistic recall seems to fail to adequately assess entity
resolution quality when the volume of data returned at the end of the entity
resolution process vastly exceeds the size of the ground truth}.
Despite encompassing all true positives, we are left with a myriad of false
positives to weed out.
This is a real problem for systems that value a result that is both accurate and
sensitive.

While in the context of our control dataset it may not seem likely that this
would raise concerns, consider that for practical applications the size of the
ideal result is often several orders of magnitude smaller than the size of the
input.
Taking into account other factors (like data cleanliness, for example), we see
that the odds of this situation occuring is not negligible at all.
Large entity resolution results are more likely to be obtained from large input
data sets than from smaller data sets, after all.

Therefore, while the input data set size certainly matters, the entity
resolution model also matters, as probabilistic recall clearly shows us.
Note that this phenomenon is dependent solely on the entity resolution model and
not on using a specific entity resolution solution to highlight it.
For other algorithms the same phenomenon may be observed depending on their
specific configuration.
Pairwise metrics seem to provide a good treatment for this phenomenon.

\subsubsection{Cluster Metrics}\label{subsubsec:Cluster Results}

Cluster metrics attempt to show how close two partitions over the same set are.
This family of metrics borrows naming from probabilistic metrics even though
the aspects being measured are quite different.
We see a plot of these cluster metrics over the `miny-buy' data set in
Figure~\ref{fig:mini-alg-cluster}.

\begin{figure}[htbp]
    \centering
    \captionsetup{justification=centering}
    \includegraphics[width=\columnwidth]{mini-buy-algebraic-cluster}
    \caption{Algebraic cluster metrics on `mini-buy' dataset}\label{fig:mini-alg-cluster}
\end{figure}

The optimal configuration of the algorithm when balance between precision and
recall is required still consists of $0.15 \le t \le 0.38$.
We also notice that cluster metrics point out the lack of recall when \textit{t}
nears zero even more than the pairwise metrics.
This is due to singletons having a larger influence over this metric when the
sizes of the partition classes are larger (such as when duplicates abound).
Cluster recall also decreases for $0.38 \le t \le 0.58$ much more pronouncedly
than probabilistic recall or pairwise recall when partition classes are larger.

Cluster precision also has an interesting relationship with the other precision
metrics.
Take, for example, the similarity between the curves of the probabilistic and
pairwise precision and notice that the cluster precision curve is quite
different even on our control data set show in Figure~\ref{fig:mini-alg-cluster}.
As the number of singletons increases when we get fewer and fewer true positives
in the matching phase of the entity resolution process, the cluster precision
naturally drops.
In fact, we might say that \textbf{cluster precision behaves as a complement to the
other precision metrics, having higher values when the other precision metrics
have low values and lower values when the other precision metrics have high
values}.
This relationship between the three precision metrics looks like a good
candidate for our \textbf{second invariant condition}.

\subsubsection{Clustering Indexes}\label{subsubsec:Clustering Indexes}

Clustering indexes are, just like pairwise and cluster metrics, means to
determine the similarity of two partitions over a set.
The plots of the three indexes we want to study are in Figure~\ref{fig:mini-alg}.

\begin{figure}[htbp]
    \centering
    \captionsetup{justification=centering}
    \includegraphics[width=\columnwidth]{mini-buy-algebraic-main}
    \caption{Clustering indexes on `mini-buy' dataset}\label{fig:mini-alg}
\end{figure}

We observe that the plot of the Adjusted Rand Index indicates both the desirable
and the undesirable values of \textit{t}.
We note that this index rates the configuration $0.76 \le t \le 0.81$ almost as
good as our running leader, $0.15 \le t \le 0.38$.
By following its plot, we should stay clear of values of \textit{t} from other
intervals.
It does a good job at identifying many of the problems spotted by precision and
cluster metrics.

By comparison with the ARI, the Rand Index is not very informative.
It signals that recall is not perfect for values of \textit{t} nearing zero and
gives its highest score for a configuration of $0.76 \le t \le 0.81$.
However, it also gives high scores for values of \textit{t} where all the other
metrics do not.

Note that even though both the Rand Index and its adjusted version are pairwise
metrics, they display plots that are more similar to the plot of the cluster
metrics.

The Talburt-Wang index's plot is a cluster-based indicator and it exhibits a
plot that has a similar shape to the ARI and the cluster metrics, albeit less
pronounced.
Compared to the ARI, it does not reveal $0.76 \le t \le 0.81$ as an interesting
configuration.

We might be able to extract a third invariant condition from the relationship
between the Talburt-Wang Index, the ARI and cluster metrics.
All of these metrics indicate the same neighbourhoods for favourable
configurations of the entity resolution algorithm.
The conjecture here is that \textbf{the best generic configuration balanced
between precision and recall of an entity resolution algorithm will be found
within the agreement area between ARI on one side and TWI or cluster metrics on
the other}.
The configuration obtained this way will be very good regardless of conditions
related to the data.
In our specific case, this means that we should expect to get the best results
using our \texttt{ppjoin} algorithm with a configuration $0.18 \le t \le 0.38$.

\subsection{Outcomes from Benchmark Datasets}\label{subsec:Benchmark Datasets}

So far we have seen evidence on the controlled miniature dataset that each
theoretical model provides a lens through which we can interpret various aspects
of an entity resolution task's qualitative performance.
We have stated a few potential invariant conditions that seem to be related to
the entity resolution model more than to anything else.
By running the experiment on benchmark data we want to verify whether these
conditions are related to some attributes of the data in our control data set.

The first setup we try uses the `Abt-Buy' data set.
The plots with performance metrics for our experiment are available in Figures~\ref{fig:abt-buy-fsm-main},
~\ref{fig:abt-buy-algebraic-pairwise},~\ref{fig:abt-buy-algebraic-cluster}
and~\ref{fig:abt-buy-algebraic-main}.

\begin{figure*}[htbp]
    \begin{minipage}{0.24\textwidth}
        \centering
        \includegraphics[width=\textwidth]{abt-buy-fsm-main}
        \caption{Abt-Buy statistical metrics.}\label{fig:abt-buy-fsm-main}
    \end{minipage}
    \begin{minipage}{0.24\textwidth}
        \centering
        \includegraphics[width=\textwidth]{abt-buy-algebraic-pairwise}
        \caption{Abt-Buy pairwise metrics.}\label{fig:abt-buy-algebraic-pairwise}
    \end{minipage}
    \begin{minipage}{0.24\textwidth}
        \centering
        \includegraphics[width=\textwidth]{abt-buy-algebraic-cluster}
        \caption{Abt-Buy cluster metrics.}\label{fig:abt-buy-algebraic-cluster}
    \end{minipage}
    \begin{minipage}{0.24\textwidth}
        \centering
        \includegraphics[width=\textwidth]{abt-buy-algebraic-main}
        \caption{Abt-Buy clustering indexes.}\label{fig:abt-buy-algebraic-main}
    \end{minipage}
\end{figure*}\label{abt-buy}

The invariant conditions proposed so far all hold true.
The probabilistic recall is still exceedingly high when the threshold is nearing zero and the
pairwise metrics do a good job of showing that.
Next, the cluster precision is complementary to the pairwise and probabilistic
precision.
Finally, we find the best balanced configuration between precision and recall of
$t=0.28$ is well within the target interval we envisioned at the end of the
Subsection~\ref{subsec:Control Data} while the ARI, TWI and the cluster metrics correlate.

We move on to the `Amazon-Google Products' data set~\cite{vldb2010} and observe
the results in Figures~\ref{fig:amazon-googleproducts-fsm-main},~\ref{fig:amazon-googleproducts-algebraic-pairwise},~\ref{fig:amazon-googleproducts-algebraic-cluster} and~\ref{fig:amazon-googleproducts-algebraic-main}.

\begin{figure*}[htbp]
    \begin{minipage}{0.24\textwidth}
        \centering
        \includegraphics[width=\textwidth]{amazon-googleproducts-fsm-main}
        \caption{Amazon-Google statistical metrics.}\label{fig:amazon-googleproducts-fsm-main}
    \end{minipage}
    \begin{minipage}{0.24\textwidth}
        \centering
        \includegraphics[width=\textwidth]{amazon-googleproducts-algebraic-pairwise}
        \caption{Amazon-Google pairwise metrics.}\label{fig:amazon-googleproducts-algebraic-pairwise}
    \end{minipage}
    \begin{minipage}{0.24\textwidth}
        \centering
        \includegraphics[width=\textwidth]{amazon-googleproducts-algebraic-cluster}
        \caption{Amazon-Google cluster metrics.}\label{fig:amazon-googleproducts-algebraic-cluster}
    \end{minipage}
    \begin{minipage}{0.24\textwidth}
        \centering
        \includegraphics[width=\textwidth]{abt-buy-algebraic-main}
        \caption{Amazon-Google clustering indexes.}\label{fig:amazon-googleproducts-algebraic-main}
    \end{minipage}
\end{figure*}\label{amazon-google}

The picture is very similar to that painted in the experiment run on top of the
`Abt-Buy' dataset.
The probabilistic recall is very high due to an imbalance between the entity
resolution result size and the ground truth size.
The pairwise metrics are not prone to this evaluation fallacy and this is shown.
Cluster precision complements pairwise and probabilistic precision very clearly.
And, finally, the best score for entity resolution that is balanced between
probabilistic precision and recall is obtained for $t=0.3$ which is well within
the target interval.

Finally, we look at the `DBLP-ACM' benchmark dataset.
The plots we obtained after running the experiment on this dataset are available
in Figures~\ref{fig:dblp2-acm-fsm-main},~\ref{fig:dblp2-acm-algebraic-pairwise},~\ref{fig:dblp2-acm-algebraic-cluster} and~\ref{fig:dblp2-acm-algebraic-main}.

\begin{figure*}[htbp]
    \begin{minipage}{0.24\textwidth}
        \centering
        \includegraphics[width=\textwidth]{dblp2-acm-fsm-main}
        \caption{DBLP-ACM statistical metrics.}\label{fig:dblp2-acm-fsm-main}
    \end{minipage}
    \begin{minipage}{0.24\textwidth}
        \centering
        \includegraphics[width=\textwidth]{dblp2-acm-algebraic-pairwise}
        \caption{DBLP-ACM pairwise metrics.}\label{fig:dblp2-acm-algebraic-pairwise}
    \end{minipage}
    \begin{minipage}{0.24\textwidth}
        \centering
        \includegraphics[width=\textwidth]{dblp2-acm-algebraic-cluster}
        \caption{DBLP-ACM cluster metrics.}\label{fig:dblp2-acm-algebraic-cluster}
    \end{minipage}
    \begin{minipage}{0.24\textwidth}
        \centering
        \includegraphics[width=\textwidth]{abt-buy-algebraic-main}
        \caption{DBLP-ACM clustering indexes.}
        \label{fig:dblp2-acm-algebraic-main}
    \end{minipage}
\end{figure*}\label{dblp2-acm}

Probabilistic recall is almost twice as high as pairwise recall, even though
they have the same plot.
Cluster precision is again very high where probabilistic precision and
pairwise precision are very low.
The configuration $t=0.32$ which has the highest $F_1$ score is still within the
interval predicted on our control data set.

Running the experiments has also proven to be a source of insight into the
computational expenditure surrounding entity resolution itself as well as the
calculation of entity resolution metrics.
This prompts a closer look into finding a balance between high quality outcomes
and the quantitative performance of the entity resolution solution.

    \section[conclusion]{Conclusions}\label{sec:conclusions}

    Entity resolution is a complex task that still holds many unknowns.
    When evaluating entity resolution results in practice it often happens that
    the results on control data sets are not the same as the results obtained
    on `real-world' data.
    In the end, those results speak about certain expectations that stem from
    the way we interpret data.

    We set out on a journey to find out whether entity resolution models might
    provide an answer.
    Entity resolution models can be understood in many ways.
    Our own perception is that entity resolution models are simply theoretical
    perspectives over the practical act of determining whether two entity
    references refer to the same real-world entity.

    Depending on the entity resolution model, we saw that we may have different
    data structures to take into account when running an entity resolution
    process: from lists of pairs to partitions over a set.
    Depending on the perspective and the data structures we choose, we can say
    different things about the data we analyse.
    We posited that we can even change our point of view from one perspective to
    another without any data loss.

    Lastly, our experiment setup was designed to find invariant conditions that
    might help reduce the time of picking a good entity resolution algorithm
    configuration by leveraging entity resolution models.
    One such condition is that \textbf{probabilistic recall is always going to
    be distorted when the size of the entity resolution result is much greater
    than the size of the ground truth.}
    Luckily, pairwise recall is the perfect antidote for this distortion.
    Probabilistic metrics might be cheap and easy to use, but use pairwise
    metrics when in doubt.

    Another invariant is that \textbf{cluster precision is a good counterweight
    to probabilistic and pairwise precision.}
    This leads to cluster metrics forming a harmonious complement to
    probabilistic metrics, pairwise metrics and to the Adjusted Rand Index (ARI)
    which is also pair-based.
    
    This leads us to our third invariant: \textbf{the Talburt-Wang Index and the
    cluster metrics on one side and the Adjusted Rand Index on the other usually
    agree only when there is a good balance between precision and recall.}
    That means that the configuration value intervals where these metrics agree
    on a control data set provides us with a good estimation of the best
    configuration options for running an entity resolution algorithm.
    These configurations seem to hold true irrespective of the data we run
    entity resolution on.

    % \section[future]{Future Work}\label{sec:future}

    The current paper covers only two models and there are more to cover.
    Existing models such as the one proposed by the Stanford Entity Resolution
    Framework\cite{Ben2009Swoosh} or graph theoretical models used frequently in
    entity alignment tasks are prime candidates.
    On the other hand, we can also propose new models based on recent
    developments in the field of machine learning and artificial intelligence.

    From an experimental perspective, we can expand our experiment base using
    more and more algorithms to find out whether the conjectures outlined in
    the current paper are worth proving formally.
    The other aspect that is left unresolved is the claim that probabilistic
    recall provides a distorted interpretation over the entity resolution result
    when the result size is significantly larger than the ground truth.
    Albeit challenging, experiments to verify this claim can and should be
    designed.

    Lastly, another significant dimension of entity resolution refers to the
    quantitative measurement of performance.
    Entity resolution is usually an expensive process in terms of computational
    resources.
    A solution that balances well the outcome quality with performance is
    desirable.


    \balance

    \bibliographystyle{ieeetr}
    \bibliography{er-general,er-related-work,er-additional-references,er-software}
\end{document}
