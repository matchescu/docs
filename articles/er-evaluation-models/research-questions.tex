In scientific literature, entity resolution has been attributed numerous
characteristics\cite{Tal11,Pap19}.
In the context of this paper, the following stand out:

\begin{itemize}
    \item\textit{A concern for a real-world entity}, for if the task were
    not about a something from our world, there couldn't be any resolution;
    \item\textit{No implication as to the used method}, because if we
    stipulated a certain way to perform entity resolution, we might not be
    able to;
    \item\textit{The representation of information is computer friendly},
    because if it weren't, it might not be possible to study the task by
    means of computer science.
\end{itemize}

The first two statements suggest that entity resolution can be performed
using any informational support.
Simply imagine collecting written notes about the same real-world object to
create an informational profile of that object.
Entity resolution becomes relevant to computer science when it involves
processing information about real-world entities through computational
methods, rather than directly interacting with the entities
themselves~\cite{Tal11}.
This approach is characterized by the constraint of handling data that is
computable, indicating that entity resolution tasks executed on computers
do not have direct access to the entities they aim to resolve~\cite{Chen09}.

Whenever we talk about measuring the performance of an entity resolution
task, we must note that we are confined to the information that is available
to the computer running the task.
On the other hand, as humans we judge the quality of the entity resolution
task's result in relation to the real-world information available to us, which
usually includes the real-world object itself.
This produces a gap between what a computer can provide us with and the human
observer's expectation, a gap which is starting to garner attention~\cite{wang2022realideal}.
Coincidentally we are nearing a century since it has become apparent that the
observer is key in our physical universe~\cite{schrodinger1926}.
Perhaps we can find inspiration for understanding entity resolution quality in
Schrodinger's insight.

There are multiple models that describe entity resolution.
Each theoretical model's job is to describe something (in our case, entity
resolution) using existing notions.
Some use probabilities, some use linear algebra and some use graphs.
All of them describe entity resolution completely.
The depth at which they are able to describe entity resolution depends on the
allowances of the tools they use.
Entity resolution is described better by richer data structures and more complex
processes.
To this end, entity resolution models make up two main families: matching models
and clustering models.
Our intent is to relate two models from each family.

In this context, we may find conditions that originate in the entity resolution
model itself rather than in reality, the data provided to the computer or the
entity resolution solution.
These conditions might be abstract, but they might also provide valuable
insight to the human observer that analyses an entity resolution outcome.
Our \textbf{first goal is to find out whether such invariant conditions might
actually exist}.

Next, another gap presents itself between the generic solutions provided by
scientific literature or programming libraries and the specific needs derived
from a context of the practitioners actually implementing entity resolution
systems.

Historically, entity resolution measurements owe much to the field of
information retrieval and to statistics.
Implementing entity resolution systems in practice, we often notice that
using popular metrics borrowed from these fields (such as the $F_1$ score) we
end up overlooking some essential aspect.
For example, in anomaly detection problems we are primarily interested in
sensitivity, not the balanced outlook the $F_1$ score provides.
Given that anomaly detection represents a whole class of solutions, one might
argue that the need of highly sensitive systems is not that specific.
Which makes this point all the more relevant.

In scientific literature and programming, generic solutions are preferred to
specific ones.
This bias towards generic solutions makes it hard for practitioners to rank
entity resolution solutions by their usefulness in the context of the task
at hand (e.g anomaly detection).
Choosing an entity resolution solution remains a necessary step in
implementing an entity resolution system.
The longer it takes to find one, the higher the implementation costs.
Also, choosing a complex system is often a decision that's difficult to
change later and this contributes to the risk of making a poor choice.

Papers have been published on the biases inherent in certain metrics
and how they influence judging entity resolution outcomes in a specific
context~\cite{Goga2015}.
The few systems out there that do provide some form of ranking entity
resolution solutions tend to use balanced metrics~\cite{papwithcode2019}.
Our previous example shows that being balanced is actually just another bias
which may or may not be useful, depending on context.

All this goes to show that the myriad ways~\cite{hitesh2012} in which we
evaluate entity resolution quality suggests that we haven't found the ideal
perspective for the entity resolution problem.
It might also show that it will take us a long time to find it.
From this point of view, it seems logical to look at \textbf{how some of the
existing metrics might complement one another to enhance our view over an entity
resolution result}.

As previously stated, entity resolution quality evaluation owes much to
information retrieval and to probability theory.
Additionally, metrics originally designed to aid in graph theory problems have
become useful, too~\cite{hitesh2012,Kon19}.
Nevertheless, a cursory review of what is usually being measured in lieu of
judging entity resolution quality~\cite{fever2009,Men10,Goga2015} shows an
important interest for evaluating the quality of entity matching.
At this point we refer to entity matching as simply finding out how similar two
data are by using various comparison techniques.
This may be due to a lingering confusion between entity matching and entity
resolution~\cite{Tal11}.

Literature~\cite{Tal11} describes other perspectives available to us regarding
entity resolution, besides seeing it as a matching problem.
The entity resolution quality metrics that measure aspects which go beyond
matching~\cite{Men10,tal2007algebraic} provide us with the pragmatic means to
see these alternate views.
On the other hand, entity resolution models\cite{Ben2009Swoosh,Tal11,fs1969}
provide us with the abstract perspective.
We interpret this to mean that certain metrics are linked to certain entity
resolution models.
With this view in mind, our final question is \textbf{whether it's possible to
interpret the same entity resolution result using two different entity
resolution models}.
