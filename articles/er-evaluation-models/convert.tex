
As stated in Subection~\ref{subsec:Fellegi-Sunter Model}, one of the misgivings of the Fellegi-Sunter model is that
it captures relatively well the matching aspect of entity resolution without
doing as good a job for its clustering aspect.
The algebraic model allows a more complete perspective over how entity
resolution works by supporting measures of both matching (pairwise metrics)
and clustering.
One of the questions we asked initially is whether we could transition from
one model to the other to enjoy the more complete overview offered by the
algebraic model.

The answer to this question depends on whether we can automatically convert
sequences of pairs of items to clusters.
Then, if that is possible, the question becomes whether from a set of pairs
of items from an input set we can obtain a partition over the $Ref$ domain
itself.

The idea is to consider pairs of items as being edges in a graph.
Taking inspiration from Kruskal's algorithm~\cite{kruskal1956} to construct
partitions, we are using a slightly altered version of the first algorithm
presented by Hopcroft and Ulman~\cite{hopcroft1973set}.
The algorithm uses the Union-Find data structure~\cite{unionfind1964}.
This data structure was originally used to keep track of a partition of a set,
where partition classes could be merged.
It allows answering whether two elements of the set are in the same partition
class or not.

Our version of the algorithm takes as input a set of items and an iterable
sequence of pairs of items from that set.
The Union-Find data structure is initialized with all items in the set.
Each pair it processes represents a pair of items that must be placed in the
same partition class.
The set merging algorithm accomplishes this.

We can see that all of the items in the set are used and none are discarded.
The Union-Find data structure produces stable partition classes of items.
In case an item is part of two partition classes, these classes are merged.
The algorithm leaves the items that do not appear in the list of pairs passed as
input in singleton partition classes.
Therefore the Union-Find data structure produces a partition over the input
set of items.

This algorithm can be applied to meet our needs by leveraging the input domain
$Ref$ and the representation of the entity resolution result in the F-S model.
Recall that $Ref$ is always defined as a set of entity references.
We defined the output of the F-S model as a set of pairs of entity
references.
The ground truth expressed according to the F-S model and the $Ref$ set of
entity references are the input parameters of the Union-Find data structure.

The ground truth always contains the ideal entity resolution output
regardless of the mathematical model.
It cannot be incomplete or faulty, by definition.
Therefore, when we apply the Union-Find data structure to the ground truth
it cannot return anything less than a ground truth because it processes all
the data in $Ref$ and the partition it creates over $Ref$ is stable (i.e
given the input sequence of pairs it cannot produce two partitions over $Ref$
on two separate occasions).

If we have a ground truth expressed as an iterable sequence of matching
pairs and we know the input $Ref$ domain that is used for entity resolution,
we obtain an ideal partition induced by entity resolution over $Ref$.
