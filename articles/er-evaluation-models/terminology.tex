Throughout this paper the term ``entity'' is reserved for real-world items.
We need new terms for the computer-friendly representations of entities.
We start by saying that entity resolution operates on data sources.

\begin{defn}
    A \textit{data source} represents a sequence of decoded messages
    that originate somewhere and can be processed by a computer program that
    reads them over a communication channel\footnote{The terms `decoded', `message' and `communication channel' refer to concepts
    defined in information theory~\cite{ash2012it}.}.
\end{defn}



An entity resolution task needs at least one data source to operate on.
When the relevant information is split across multiple data sources, these
can be merged into a single source.
In the same way, a single data source can be split into multiple data
sources.
Data sources may be bounded (files or databases) or unbounded (streams).
For the purposes of this paper, these attributes of data sources as well as
their number are inconsequential.

In practical terms, entity resolution is a computer process that operates
over a data source.
In the process of entity resolution, we use specific notions to refer to the
various levels in which data is organised.

\begin{defn}
    An \textit{attribute} is information about an entity that has a certain
    meaning in a context given by a frame of reference and certain rules of
    interpretation.
\end{defn}

Attributes by themselves are not enough to describe an entity.
For example, `red' fully describes an entity only if we are looking for
colors.
If we are looking for fruit, `red' is only one of several attributes that
help describe fruit.
We might be interested whether the fruit is `sour' or `sweet' as well as
whether it is `small', `medium' or `large'.

\begin{defn}
    An \textit{entity reference} is a collection of attributes that refer
    to a real-world entity which can be formed by following an organizing
    principle (or order) of the data source where the attributes are all
    located.
\end{defn}

The organizing principle of a data source is simply an order that
facilitates processing information in that data source.
A CSV file has the organizing principle that attributes are separated by
commas.
Relational databases organize data in tables, columns and rows.
Other storage organizes data in structured, semistructured or even
unstructured ways (such as text).
As an example of unstructured organization, named entity disambiguation uses
the same organizing principle as named entity recognition built around
mentions.
When an entity resolution task processes a data source, it loads
\textit{references}\cite{Ben2009Swoosh} to entities by taking advantage of
the way that data source is organized.

We shouldn't confuse the organizing principle of a data source with its
structure, though.
A table record in a database might store multiple attributes of the same
entity in the same column.
The entity reference will be constructed by extracting all attributes, not
by dumbly considering each column to be an attribute.
Entity reference extraction is also selective by nature.
For entity resolution tasks that value geolocation above other things, the
extracted references might skip over data that is not about geography.
Similarly, in tasks that identify people, entity references will be
comprised from attributes that are useful in identifying a person or a
company.
All this speaks to how the organizing principle of a data source is
extrinsic to the data source and linked to the entity resolution's purpose.

A final note on entity references is that attributes that each reference is
bound to a data source.
The definition above states this by mandating that all attributes comprising
an entity reference are read from the same data source.

So far we have dealt with structure, now it is time to deal with meaning and
purpose.    
Let's suppose that we are gathering information about colors.
A data source represents colors as `red', `green' or `blue'.
Another one represents them as \texttt{0xCC0000}, \texttt{00CC00} or
\texttt{0x0000CC}.
The organizing principle instructs the entity resolution task to construct
entity references with a single attribute.
It does not provide any semantic interpretation rules for the data.

\begin{defn}
    A \textit{trait} is a semantic rule that guides the entity resolution
    task in recognizing the organizing principle of each information source
    that it uses as input.
\end{defn}

Each entity resolution task is defined by specific objectives it aims to
achieve.
For every distinct task, there are key rules that guide the interpretation
of the available data, emphasizing certain aspects over others.
In our example, entity resolution focuses exclusively on identifying colors
in different data sources.
To achieve this goal, the traits of the entity resolution task instance
might be specified, trained or inferred from a knowledge base to recognize
various representations of colours.
While we might think of traits as being a part of a generic data extraction
process such as scraping, traits in the sense described here are actually
integral to the entity resolution process itself.

The notion that is the most similar to that of a trait is the pattern
recognition notion of a `feature':
\textit{an individual measurable property or characteristic of a
phenomenon}\cite{bishop2006pattern}.
The mechanics of how traits and features function is near identical.
The difference between a trait as defined here and a feature is a matter of
perspective.
Features are part of constructing a broader, objective perspective on
information.
Traits as they are defined here are concerned with the contextual meaning.

Pragmatically, traits can be viewed as algorithms that extract information
according to the subjective goal of the entity resolution task.
In their simplest form, traits might be configuration parameters for an
entity resolution system.
But they can also be algorithms that are part of the a system's code.

For example, a trait called `profitable' that is supposed to specify whether
a company earns more than it spends is different from the typical physical
representation of profit and loss as two separate decimal values present
in information organized in some way or another.
The way we would handle this discrepancy while extracting entity references
is by having an algorithm that outputs a truth value when gains exceed
spending.

Another example of a trait, is a configuration value that specifies the
attributes that should be used when comparing two entity references.
This trait might optimize certain attributes of the entity references for
comparison.

To avoid confusing traits and attributes, it is easy to think about traits
as not being a part of the data source, but rather as a part of the entity
resolution task implementation.
Entity resolution tasks extract attributes that they assign to entity
references based on the traits specific to each task.

Traits are a necessary abstraction if we want the same entity resolution
task to be able to process more than one data representation.
They are useful for constructing individual entity references.
However, we have not defined any notion that links entity references to an
entity.

\begin{defn}
    The logical group of entity references that point to the same real-world
    entity according to the entity resolution task's parameters is known as
    the \textit{entity profile} of that real-world entity within the entity
    resolution task.
\end{defn}

It is important to understand that the entity profile and the real-world
entity are not the same.
Entity profiles are simply digital collections of entity references that are
formed because of how an entity resolution task was programmed and
configured.

By using the terms \textit{information source}, \textit{attribute},
\textit{trait}, \textit{entity reference} and \textit{entity profile} we can
finally move on to define entity resolution formally.
