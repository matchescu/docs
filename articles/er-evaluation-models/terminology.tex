Throughout this paper the term ``entity'' is reserved for real-world items.
This requires introducing new terms for the computer-friendly representations of
entities.
We start by saying that when entity resolution is performed by computers, it
uses data sources.

\begin{defn}
    A \textit{data source} represents a sequence of decoded messages
    that originate somewhere and can be processed by a computer program that
    reads them over a communication channel\footnote{The terms `decoded', `message' and `communication channel' refer to concepts
    defined in information theory~\cite{ash2012it}.}.
\end{defn}



An entity resolution task needs at least one data source to operate on.
When the relevant information is split across multiple data sources, these
can be merged back into a single source.
Similarly, a single data source can be split into multiple data sources.
Data sources may be bounded (files or databases) or unbounded (streams).
For the purposes of this paper, these attributes of data sources as well as
their number are inconsequential.
In practical terms and as far as we are concerned it should suffice to know that
entity resolution is a computer process that operates over data sources.

In the process of entity resolution, we also use specific notions to refer to
the various levels in which data is organised.

\begin{defn}
    An \textit{attribute} is information about an entity that has a certain
    meaning in a context given by a frame of reference and certain rules of
    interpretation.
\end{defn}

Attributes by themselves are not enough to describe an entity.
For example, `red' fully describes an entity only if we are looking for
colors.
If we are looking for fruit, `red' is only one of several attributes that
help describe fruit.
We might be interested whether the fruit is `sour' or `sweet' as well as
whether it is `small', `medium' or `large'.

\begin{defn}
    An \textit{entity reference} is a collection of attributes that refer
    to a real-world entity which is constructed by following an organizing
    principle (or order) of the data source containing the attributes.
\end{defn}

The organizing principle of a data source is simply an order that
facilitates processing information in that data source.
A CSV file has the organizing principle that attributes are separated by
commas.
Relational databases typically organize data in tables, columns and rows.
Other storage organizes data in structured, semistructured or even
unstructured ways (such as text).
As an example of unstructured organization, named entity disambiguation uses
the same organizing principle as named entity recognition which is built around
the notion of a `mention'.
In the process of entity resolution, it loads \textit{references} to entities
from a data source by taking advantage of the way that data source is
organized~\cite{Ben2009Swoosh}.

Even though they usually go hand in hand for well-structured data, let us not
confuse the organisation principle of information with its structure, though.
Multiple attributes of the same entity might be stored in the same column of the
same database table row, for example.
In this case, the entity reference will contain all attributes, not the whole
value determined by the row and column.

Besides not being bound to structure, entity reference extraction is also
selective.
For entity resolution tasks that value geolocation above other things, the
extracted references might skip over data that is not about geography.
Similarly, in tasks that identify people, entity references will be
comprised from attributes that are useful in identifying a person or a
company.

These remarks speak to the extrinsic nature of the organizing principle of a
data source relative to that data source.
From the standpoint of entity resolution, the organization of information in a
data source is determined by the purpose for conducting entity resolution.

Finally we note that entity references contain attributes from a single data
source.
Our definition of entity references states this explicitly to avoid confusion
regarding the provenance of entity references.

Given that information organisation is driven by purpose, we move on to
describing and defining notions related to meaning and purpose in the context of
entity resolution.
Let's suppose that we are gathering information about colors.
A data source represents colors as `red', `green' or `blue'.
Another one represents them as \texttt{0xCC0000}, \texttt{00CC00} or
\texttt{0x0000CC}.
The organizing principle instructs the entity resolution task to construct
entity references with a single attribute.
It does not provide any semantic interpretation rules for the data.

\begin{defn}
    A \textit{trait} is a semantic rule that contributes to realizing the
    organizing principle related to the purpose of a specific entity resolution
    solution over an input data source.
\end{defn}

Whenever entity resolution is performed, it is performed with purpose and
agency.
There are specific objectives involved, such as medical diagnoses, plagiarism
detection or illegal financial transaction monitoring.
Different purposes bring into existence different key rules that guide the
interpretation of the available data, emphasizing certain aspects over others.
These rules are typically implemented deliberately to serve the purpose for
performing entity resolution.

In our trivial example, entity resolution focuses exclusively on identifying
colors in different data sources.
To achieve this goal, the traits of the entity resolution task instance
might be specified, trained or inferred from a knowledge base to recognize
various representations of colours.
While we might think of traits as being a part of a generic data extraction
process such as scraping, traits in the sense described here are actually
integral to the entity resolution process itself.
In practice, generic data collection and extraction usually come into play when
constructing the input data sources used in the entity resolution process.
On the other hand, traits come into play within entity resolution, when
constructing entity references.

The most similar notion to that of a trait is the pattern recognition notion of
a `feature':
\textit{an individual measurable property or characteristic of a
phenomenon}\cite{bishop2006pattern}.
The mechanics of how traits and features function is near identical.
The difference between a trait as defined here and a pattern recognition feature
is a matter of perspective.
Features are part of constructing a broader, objective perspective on
information.
Traits as they are defined here are concerned with mapping an objective
information to the entity resolution context.

Pragmatically, traits can be viewed as computing functions that process
information according to the subjective goal of the entity resolution task.
In their simplest form, traits might be configuration parameters for an
entity resolution system.
But they can also be algorithms that are part of the a system's code.

For example, a trait called `profitable' that is supposed to specify whether
a company earns more than it spends is different from the typical physical
representation of profit and loss as two separate decimal values present
in information organized in some way or another.
The way we would handle this discrepancy while extracting entity references
is by having an algorithm that outputs a truth value when gains exceed
spending.

Another example of a trait, is a configuration value that specifies the
attributes that should be used when comparing two entity references.
This trait might optimize certain attributes of the entity references for
comparison.
Some entity resolution algorithms allow configuring thresholds (such as the
Jaccard threshold~\cite{jaccard1912}) to determine an upper limit to the number
of attributes of entity references.

To avoid confusing traits and attributes, it is easy to think about traits
as not being a part of the data source, but rather as a part of a specific
implementation of entity resolution.
Entity resolution computer programs extract attributes that they assign to 
entity references based on the specific traits built into each such program.

Traits are a necessary abstraction that allows us to envision multiple entity
resolution implementations over the same data source.
These implementations would differ in their purpose.
They are useful for constructing individual entity references.

In the definition of entity resolution as the task of determining whether two
information refer to the same real-world item, entity references are the two
information.
Next, we provide the notion that defines the representation of the real-world
item throughout the entity resolution process.

\begin{defn}
    The logical grouping of entity references that point to the same real-world
    object according throughout entity resolution is known as the \textit{entity
    profile} of the real-world object.
\end{defn}

The entity profile must not be confused with the real-world object it describes.
Entity profiles are representations of the real-world object that are available
to the system performing entity resolution.

By using the terms \textit{information source}, \textit{attribute},
\textit{trait}, \textit{entity reference} and \textit{entity profile} we can
finally move on to define entity resolution formally.
