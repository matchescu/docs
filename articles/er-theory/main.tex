\documentclass[a4paper,12pt]{article}
\usepackage[T1]{fontenc}
\usepackage{array}
\usepackage{booktabs}

\begin{document}

\section{Introduction}

Entity resolution is the process of finding information that refers to the same
real-world entity. We call the people that performed this task librarians,
journalists or detectives among others. What these professionals have in
common is their ability to understand what it is they're looking for and how to
look for the information regardless of the medium where it is found.
Furthermore, they have an uncanny way for assembling their findings in a way
that makes sense to others so that it's easy for everyone to agree that the
assembled information indeed refers to the same real world entity.

We call the collection of information about a certain real-world entity its
\textit{entity profile}. It's important to distinguish between the entity
profile and the entity itself[quotation needed]. For one, the entity profile may
be represented in a great number of ways whereas the real-world entity itself is
unique in its representation in the real world (as far as we know). In theory,
an entity profile could store information encoded at the hardware level in any
number of ways. For example, the same entity profile could span across multiple
machines some of which could be ternary computers or quantum computers.

Another way to look at information representation is to deepen our understanding of a given
modality of representation. Take for instance the textual representation of the
word `orange'. The text itself can be represented using one of many text
encodings (ASCII, utf-8, unicode, etc) which are all implemented on top of the
same mathematical basis of boolean algebra. Who is to say they could not also be
represented using an entirely different mathematical basis, for instance?

The same observation that we made for text holds true for the other broad
modalities in which information can be represented. To be useful, any type of
information must be represented in whichever way a physical carrier medium
allows humans to derive meaning from the information. Meaning is important in
entity resolution because `orange' doesn't mean the same thing when used as
part of the syntagm `orange tree' as it does in the syntagm `orange is the new
black'. The representation of information is a deciding factor in how humans
derive meaning from it. Stating the obvious, entity resolution must make sense
from the human point of view and therefore entity resolution depends on how
information is represented because it depends on how humans derive meaning from
the physical information support. The other major contributor to how humans
derive meaning is the context of the information.

At its core, entity resolution concerns itself with the `features' of a
real-world entity.


\end{document}